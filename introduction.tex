\section{Introduction}
\label{ch:introduction}

%******* Introduction of the Dash technology and Cloud/Fog networks****


% ----------------------------------------------------------------------------
% What is the current context in mobile network in terms of traffic and users? 
% ----------------------------------------------------------------------------
Os serviços de streaming de vídeo representam a maior parte do tráfego da Internet. De acordo com as previsões da Cisco~\cite{cisco:forecast},
%~\footnote{Visual Networking Index: atualização global de previsão de tráfego de dados móveis. Link:~\url{http://shorturl.at/hjAZ1}. Acesso em: 29 de julho de 2019.}, 
em 2021, 70\% de todo o tráfego da Internet será dominado por streaming de vídeo. Isso inclui os serviços de vídeo atuais, bem como serviços inovadores de jogos na nuvem e futuros consoles~(por exemplo, Google Stadia), enquanto que para dispositivos móveis, essa estimativa representa 78\% de todo o tráfego de dados. Essa tendência impõe novos desafios no fornecimento de vídeos com a melhor Qualidade de Experiencia~(QoE), originalmente projetada considerando o modelo de \textit{best-effort} para transmissão de dados.

% ----------------------------------------------------------------------------
% What carries have been doing to address the increasing traffic?
% ----------------------------------------------------------------------------
% Em cenários para as futuras gerações da Internet, tecnologias de redes sem fio, como, redes 5G, mobile-edge computing~(MEC), redes definida por software~(SDN).
%Esses fatores demonstram como o futuro da demanda da Internet pode se tornar muito grande para os ISPs e os avanços tecnológicos nas telecomunicações nas quais eles dependem. 
Atualmente, os tradicionais serviços de Streaming de vídeo são projetados para distribuir conteúdo multimídia através de grandes data centers na nuvem. Esses sistemas na nuvem geralmente usam um conjunto de servidores em que o tráfego passa pelo núcleo da rede. Além disso, de modo geral, a conexão dos dispositivos é feita por usuários estáticos e por links de Internet estáveis~\cite{sitaraman:ACD2014}. Dessa forma, a criação desses sistemas resolve parcialmente os problemas de escalabilidade, disponibilidade e interoperabilidade, mas ao mesmo tempo, apresenta novos desafios~(por exemplo, maior latência e congestionamento da rede principal)~\cite{tran:wons17,ye:ITC17,taleb:JSAC18}. Vários trabalhos na literatura destacam a computação de nevoa/borda para lidar com as novas demandas de tráfego de video que estão surgindo. Onde datacenters com menos capacidade de processamento e armazenamento conseguem prover serviços/virtualização mais próximos ao usuário final. Assim, a borda da rede pode fornecer taxas de latência que a nuvem não conseguem alcançar de outra forma~\cite{gamaUCC2019, rosarioSENSORS2018}.
% ou caches especificos alocados pelo proprio administrador da rede

%, levando em consideração as novas tecnologias que estão surgindo, como, redes 5G, mobile-edge computing~(MEC), redes definida por software~(SDN)


Os serviços de streaming de vídeo têm requisitos rigorosos, como um canal de comunicação de boa qualidade, bem como um fluxo constante e ininterrupto de informações~\cite{Immich2018WinNet}.
% A Fig. 1 Mostra ao longo de cada ano o crescimento no consumo de banda, estes tipos de cenários impõem significantes desafios para a distribuição de vídeo sobre as futuras gerações da Internet.
Para acomadar esta demanda, bem como manter o QoE do usuário, grandes players como Microsoft, Apple, Adobe e Netflix adotam o paradigma de streaming adaptável sobre HTTP~(HAS)~\cite{company:dashs}. Como a maioria das soluções HAS usa a mesma arquitetura, o Motion Picture Expert Group~(MPEG) propôs um padrão chamado Dynamic Adaptive Streaming over HTTP~(DASH), no qual o player de vídeo pode escolher dinamicamente a taxa de bits de acordo com a largura de banda disponível percebida.
%Atualmente, esses conteudos multimidia utilizam serviços de Streaming Adaptativo Dinâmico sobre HTTP~(HAS), este paradigma trata o video multimídia como qualquer outro conteúdo comum da Web e o entrega em pequenos pedaços atraves do protocolo HTTP.
%Os serviços de transmissão de vídeo têem seu bitrate adaptados dinâmicamente.% As soluções HAS utilizam os protocolos HTTP na aplicação e TCP na camada de transporte como ilustrados na figura 1b. 
%em Streaming Adaptativo Dinâmico sobre HTTP~(DASH) são amplamente adotado por provedores de vídeo como Google, Netflix, Akamai HD e outros, nos quais o player de vídeo do cliente pode escolher dinamicamente o nível de taxa de bits de acordo com a largura de banda disponível percebida.
Esta solução emprega adaptação dinâmica em relação a variação das condições de rede para fornecer uma experiência de streaming sem interrupções~(ou pelo menos mais suave). Depois que um arquivo de mídia~(ou fluxo) está pronto a partir de um servidor fonte, ele pode ser transmitido em segmentos por um servidor HTTP padrão.

%HAS uses HTTP as the application and TCP as the transport-layer protocol as illustrated in Figures 1.2b and 1.3, and clients pull the data from an HTTP server. HAS solutions employ dynamic adaptation with respect to varying network conditions to provide a seamless (or at least smoother) streaming experience. Once a media file (or stream)is ready from a source, it is prepared for streaming before it is published to a standard, off-the-shelf HTTP server. The original file/stream is partitioned into segments (also calledchunks) of equi-length playback time, and multiple versions (also called representations)of each segment are generated that vary in bitrate/resolution/quality using an encoder ora transcoder (i.e., H.264, H.265, etc.). Moreover, the server generates an index file, whichis a manifest that lists the available representations including HTTP uniform resourcelocators (URLs) to identify the segments along with their availability times. During atypical HAS session, the client first receives the manifest that contains the metadata forthe video, audio, subtitles, etc., and then constantly measures certain parameters such as the available network bandwidth, buffer status, and battery and CPU levels. According tothese parameters, the HAS client repeatedly fetches the most suitable next segment amongthe available representations from the server. Table 1.1 compares the main characteristicsof the traditional streaming and HAS systems.

% ----------------------------------------------------------------------------
% What carries have been doing to address the increasing traffic?
% ----------------------------------------------------------------------------
%Em cenários para as futuras gerações da Internet, tecnologias de redes sem fio, como, redes 5G, mobile-edge computing~(MEC), redes definida por software~(SDN).
%Esses fatores demonstram como o futuro da demanda da Internet pode se tornar muito grande para os ISPs e os avanços tecnológicos nas telecomunicações nas quais eles dependem. 
%Atualmente, tradicionais serviços de Streaming de vídeo sob Demanda~(VoD) são projetados para distribuir conteúdo multimídia por grandes data centers na nuvem. Esses sistemas geralmente usam um conjunto de servidores em que o tráfego passa pel núcleo da rede; além disso, a conexão dos dispositivos é feita por usuários estáticos e por links da Internet estáveis [?]. Dessa forma, a criação desses sistemas no nível da nuvem resolve parcialmente os problemas de escalabilidade, disponibilidade e interoperabilidade, mas ao mesmo tempo, apresenta novos desafios (por exemplo, maior latência e congestionamento da rede principal) [?]. Vários trabalhos na literatura destacam a computação de nevoa/borda para lidar com as novas demandas de tráfego que estão surgindo. Onde datacenters com menos capacidade de processamento e armazenamento conseguem prover serviços/virtualização mais próximos ao usuário final. Assim, a borda da rede pode fornecer taxas de latência que a nuvem não conseguem alcançar de outra forma [?], [?].

%These two factors demonstrate how the future of Internet demand may become toooverwhelming for both ISPs and technological advancements in telecoms they rely on. Currently, the traditional VoD services are designed to distribute multimedia content by large data centers at the cloud. These systems usually use a set of servers where the traffic passes through the core network, beyond that, the devices connection are made by static end-users and Internet links stable [?]. This way, the inception of these systems at cloud-level partially solves the scalability, availability, and interoperability issues, but at the same time, introduces new ones (e.g., higher latency and core network congestion) [?]. Several works in the literature have highlighted fog/edge computing to deal with new traffic demands that are emerging. Where datacenters with less processing and storage capacity can provision virtualization/services closer to the end-user. Thus, the network edge can provide latency rates that the cloud is not able to reach otherwise [?], [?].


% ----------------------------------------------------------------------------
% What are the problems with existing networks?
% ----------------------------------------------------------------------------

%Para melhorar os serviços de vídeo, é de suma importância distribuir adequadamente os fluxos de vídeo de acordo com seus requisitos: uma infraestrutura de jogos em nuvem é um serviço interativo que precisa de atrasos reduzidos~(alguns milissegundos), enquanto uma entrega de VoD não interativa pode tolerar maior atrasos sem prejudicar a qualidade da experiência. Um gerenciamento e orquestração adequados da entrega de vídeo pela Internet é essencial para a coexistência suave de serviços de vídeo heterogêneos.

%Currently, the traditional VoD services are designed to distribute multimedia content by large data centers at the cloud. These systems usually use a set of servers where the traffic passes through the core network, beyond that, the devices connection are made by static end-users and Internet links stable [?]. This way, the inception of these systems at cloud-level partially solves the scalability, availability, and interoperability issues, but at the same time, introduces new ones (e.g., higher latency and core network congestion) [?].

%Esse novo paradigma, ao qual nos referimos como HTTP Adaptive Streaming (HAS), tratava o conteúdo da mídia como qualquer outro conteúdo comum da Web e o entregava em pequenos pedaços pelo protocolo HTTP. O HAS se tornou rapidamente a abordagem dominante para o streaming de vídeo devido à sua adoção pelos principais provedores de serviços e conteúdo. A entrega de vídeo pela Internet pública também é conhecida como streaming de vídeo over-the-top (OTT), pois o conteúdo ou o provedor de serviços de streaming geralmente é diferente do provedor de rede. O surgimento do HAS e de novos dispositivos móveis principalmente para usuários finais, com altos recursos de processamento e renderização, teve um papel fundamental no crescimento do tráfego de streaming de vídeo

%There are several works that try to solve a number of
%issues regarding the combined use of Cloud, Fog, and Edge
%computing. Generally, the existing works are based on ar-
%chitecture design and deployment issues [7]–[10]. They try
%to improve virtual machine migration, mobility adversities or
%provide smart caching for a specific type of content as well
%as to enhance video delivery with adaptive streaming and Fog
%nodes. However, the solutions presented tend to be general and
%do not take into consideration video-specific needs.

Embora muitos trabalhos de pesquisa abordem serviços de video streaming em conjunto com a computação em nuvem/nevoa. Existem aspectos pouco abordados em soluções atuais considerando uma arquitetura de streaming vídeo em multinível, geralmente, buscam diminuir a carga de tráfego e melhorar a QoE a entrega do video, além de aprimorar o offloading de maquinas virtuais.
%Eles tentam melhorar a migração de máquinas virtuais, adversidades de mobilidade ou fornecer armazenamento em cache inteligente para um tipo específico de conteúdo, além de aprimorar a entrega de vídeo com streaming adaptável e nós Fog. 
Tais soluções tendem a não levar em consideração as necessidades específicas de cada usuário, ao mesmo tempo, entrega de vídeo em multiníveis na borda pode fornecer uma taxa de latência que a nuvem não consegue prover.
%usando ferramentas de código aberto disponíveis no mercado e sequências de vídeo reais. O principal objetivo é provar que é possível criar um ambiente multicamada real para melhorar a qualidade da entrega de vídeo. 

%Taking into account the aforementioned scenarios and new
%ones yet to be revealed in the next few years there is a growing
%need for a multi-tier video delivery architecture. It could take
%advantage, at the same time, of the elastic resource pool that
%Cloud computing provides in association with the low-latency
%and high-throughput offered by the Edge computing. This
%paper aims to advance the idea of multi-tier video delivery
%using off-the-shelf open-source tools and real video sequences.
%The main goal is to prove that it is possible to build a real
%multi-tier environment to improve video delivery quality.

%, levando em consideração as novas tecnologias que estão surgindo, como, redes 5G, mobile-edge computing~(MEC), redes definida por software~(SDN).
%O paradigma SDN pode realmente ser usado para expandir as redes existentes, afastando a dependência do fornecedor e mantendo o desempenho alcançado pelo hardware dedicado? Como as soluções propostas são avaliadas e qual é a adoção efetiva dessas soluções? Como gerenciar futuras redes 5G heterogêneas, considerando diferentes tecnologias de rádio e a explosão de conexões? Como simplificar efetivamente o gerenciamento da mobilidade nas arquiteturas atuais, explorando a visão de rede centralizada da SDN?

% ----------------------------------------------------------------------------
% Which technologies can be used to improve future networks?
% ----------------------------------------------------------------------------

% ----------------------------------------------------------------------------
% How the SDN parading is contributing to current networks?
% ----------------------------------------------------------------------------

% ----------------------------------------------------------------------------
% Why and how are we contributing in this topic?
% ----------------------------------------------------------------------------

%In an attempt to answer above questions, this ongoing doctoral research project discusses how the \ac{SDN} paradigm and the OpenFlow protocol can be integrated to the existing 4G \ac{LTE} networks to provide new solutions for some of the aforementioned problems. As contributions already developed, this project first ...
 
Este projeto tem como objetivo principal desenvolver mecanismos de entrega de vídeo baseada em DASH confiável e de alta qualidade para ser usada em ambientes de cidades inteligentes [1, 2]. Tais mecanismos propostos aproveitarão de várias tecnologias relacionadas à rede, como Cloud, Fog e Edge Computing, a fim de além de posicionamento inteligente de serviços em uma arquitetura multinível. %A Figura 1 mostra, no lado esquerdo, uma arquitetura de rede de várias camadas, composta por um conjunto heterogêneo de dispositivos e aplicativos usando recursos de computação distribuídos por meio de uma tecnologia de comunicação de acesso múltiplo, como 5G e WiFi. Este projeto propõe estender o streaming de vídeo DASH para oferecer suporte à conectividade multipath simultânea [3, 4].

% ----------------------------------------------------------------------------
% How is this document organized?
% ----------------------------------------------------------------------------
Este documento está organizado da seguinte forma: \autoref{ch:background} apresenta
alguns conceitos basicos de Video Streaming Adaptativo e arquiteturas de multiníveis na Névoa/Nuvem. Em seguida, \autoref{ch:related-work} fornece uma visão geral dos trabalhos relacionados. Focando em trabalhos de Video Streaming em redes multiníveis utilizando soluções em DASH para melhorar gives an overview of several proposals foro QoE dos usuśarios da rede, comparando alguns trabalhos relacionados, destacando e comparando os principais desafios neste tópico. \autoref{ch:developed} traz o trabalho realizado até o momento, o que ajudará no desenvolvimento das contribuições propostas \autoref{ch:proposal}. Finalmente, \autoref{ch:remarks} fecha o documento com algumas considerações finais.
 
%This document is organized as follows: \autoref{ch:background} presents some
%background concepts on \ac{SDN} and \ac{LTE} architectures. Then,
%\autoref{ch:integration} gives an overview of several proposals for \ac{SDMN}.
%It focuses on the use of \ac{SDN} solutions in the backhaul and core network,
%comparing some related works and highlighting major open challenges in the
%topic. \autoref{ch:developed} brings the work carried out so far, which will
%assist the development of the proposed contributions that are detailed in
%\autoref{ch:proposal}. Finally, \autoref{ch:remarks} closes the document with
%some final remarks.

