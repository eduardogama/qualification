\documentclass[11pt, letterpaper]{article}
\usepackage[utf8]{inputenc}
\usepackage[portuguese]{babel}
\usepackage[in]{fullpage}
\usepackage[pdftex]{graphicx}
\graphicspath{{./img/}{./graphs/qos/}{./graphs/adm/}}
\usepackage{subfig}
\usepackage{amssymb}
\usepackage{setspace}
\usepackage{multirow}
\usepackage{rotating}
\usepackage[hidelinks]{hyperref}
\usepackage{url}
\usepackage{xcolor}
\usepackage{colortbl}
\usepackage{acronym}
\usepackage[square,numbers]{natbib}
\usepackage{logo-ic}
\usepackage{booktabs}

\usepackage{lscape}
\usepackage{amsmath}

\definecolor{lightgray}{gray}{0.75}

\usepackage{csquotes}


\newcommand{\TODO}[1]{{\color{red} #1}}
\newcommand{\x}{$\bullet$}
\newcommand{\m}{$\checkmark$}

% Hyperref configuration
\hypersetup {
  pdftitle={Doctoral Qualifying Exam -- Eduardo de Souza Gama},
  pdfauthor={Eduardo de Souza Gama},
  pdfcreator={pdfTeX 3.14159265-2.6-1.40.16 (TeX Live 2019)},
  pdfsubject={Doctoral Qualifying Exam - IC/UNICAMP},
  pdfkeywords={},
}

\begin{document}

\def\sectionautorefname{Section}
\def\subsectionautorefname{Section}
\def\subsubsectionautorefname{Section}
\def\figureautorefname{Figure}
\def\subfigureautorefname{Figure}
\def\tableautorefname{Table}

\sloppy

\pagenumbering{roman}
\thispagestyle{empty}

% Logos and names
\begin{center}
  \begin{tabular}{ccc}
    \raisebox{-.5\height}{\includegraphics[width=2.2cm]{logo-unicamp}}
    &
    \begin{minipage}{.6\textwidth}
      \centering
      \textbf{Universidade Estadual de Campinas} \\
      \textbf{Instituto de Computação} \\
    \end{minipage}
    &
    \raisebox{-.5\height}{\scalebox{1.11}{\LogoIcUnicampWithName}}
  \end{tabular}
\end{center}
  
\hrule
\vspace{1.5cm}

\begin{center}
  {\large \textbf{Exame de Qualificação de Doutorado}} \\
  \vspace{1cm}
  {\Large \textbf{Distribuição de Vídeo Multinível para Cidades Inteligentes}} \\
  \vspace{1cm}
      \begin{tabular}{rl}
          \textbf{Candidato}:  & Eduardo de Souza Gama \\
          \textbf{Orientador}:    & Luiz Fernando Bittencourt, D.Sc. \\
          \textbf{Co-orientador}: & Roger Immich, D.Sc. \\
      \end{tabular}
  \vspace{0.5cm}

  \begin{abstract}%
  \vspace{0.2cm}
  \onehalfspacing
O streaming de vídeo baseado no Dynamic Adaptive Streaming over HTTP~(DASH) é amplamente adotado por provedores de vídeo como Google, Netflix, Akamai HD e outros, onde o reprodutor de vídeo do cliente pode escolher dinamicamente o nível de taxa de bits de acordo com a largura de banda disponível percebida.
Ao mesmo tempo, os serviços de streaming de vídeo representam a maioria do tráfego da Internet e, de acordo com as previsões da Cisco, em 2021 70\% de todo o tráfego da Internet será dominado pelo streaming de vídeo. Isso inclui serviços de vídeo atuais, bem como serviços inovadores, como jogos na nuvem e futuros consoles~(por exemplo, Google Stadia), enquanto que para dispositivos móveis essa estimativa representa 78\% de todo o tráfego de dados móveis. Para acomodar o tráfego de vídeo, uma boa arquitetura no nível da nuvem resolve parcialmente alguns problemas relacionados à transmissão ao vivo e aos serviços Video sob Demanda~(VoD). No entanto, um serviço de nuvem centralizado apresenta alguns problemas, como maior latência e congestionamento da núcleo da rede. Portanto, para melhorar os serviços de vídeo, é de suma importância distribuir adequadamente os fluxos de vídeo de acordo com seus requisitos: uma infraestrutura de jogos em nuvem é um serviço interativo que precisa de atrasos reduzidos~(alguns milissegundos), enquanto uma entrega de VoD não interativa pode tolerar maior atraso sem prejudicar a qualidade da experiência. Um gerenciamento e orquestração adequados da entrega de vídeo pela Internet é essencial para a coexistência suave de serviços de vídeo heterogêneos. Este projeto propõe o uso da hierarquia de névoa/nuvem para projetar um streaming de vídeo DASH cooperativo em Smart Cities, implantando serviço de cache para oferecer Qualidade de Experiência aprimorada~(QoE) para usuários finais.

% Video Streaming based on Dynamic Adaptive Streaming over HTTP~(DASH) has been widely adopted by video providers such as Google, Netflix, Akamai HD and others, where the client-side video player can dynamically choose the bitrate level according to the perceived available bandwidth. 
% At the same time, video streaming services represent the majority of the internet traffic, and according to Cisco forecasts, in 2021 70\% of all internet traffic will be dominated by video streaming. This includes current video services as well as innovative services such as cloud gaming and future consoles (e.g. Google Stadia), whereas for mobile devices this estimate represents 78\% of all mobile data traffic. To accommodate video traffic, a good cloud-level architecture partially solves some issues related to the live stream and Video on Demand~(VoD) services. However, a centralized cloud service introduces some issues such as higher latency and core network congestion. Therefore, to improve video services, it is of paramount importance to properly distribute video streams according to their requirements: a cloud gaming infrastructure is an interactive service that needs reduced delays (a few milliseconds), while a non-interactive VoD delivery can tolerate higher delays without impairing quality of experience. A proper management and orchestration of video delivery over the Internet is core to the smooth co-existence of heterogeneous video services. This project proposes the use of edge/cloud hierarchy to design a cooperative DASH video streaming in Smart Cities, deploying cache service to offer improved Quality of Experience~(QoE) for end-users. 

  \end{abstract}
  \vspace{0.3cm}

  Dezembro, 2019 %\today
\end{center}


\newpage
\tableofcontents

\newpage
\section*{List of Acronyms}

\begin{acronym}[CSMA/CA]
	\itemsep0.5pt
  \acro{3GPP}     {3rd Generation Partnership Project}
  \acro{AP}       {Access Point}
  \acro{API}      {Application Program Interface}
  \acro{AS}       {Access Stratum}
  \acro{C-RAN}    {Cloud RAN}
  \acro{CoMP}     {Coordinated MultiPoint}
  \acro{DMM}      {Distributed Mobility Management}
  \acro{DSCP}     {DiffServ Code Point}
  \acro{E-RAB}    {E-UTRAN Radio Access Bearer}
  \acro{E-UTRAN}  {Evolved Universal Terrestrial Radio Access Network}
  \acro{eNB}      {Evolved Node B}
  \acro{EPC}      {Evolved Packet Core}
  \acro{EPS}      {Evolved Packet System}
  \acro{GBR}      {Guaranteed Bit Rate}
  \acro{GPRS}     {General Packet Radio Service}
  \acro{GTP}      {GPRS Tunnelling Protocol}
  \acro{HetNet}   {Heterogeneous Network}
  \acro{HSS}      {Home Subscriber Server}
  \acro{HTTP}     {HyperText Transfer Protocol}
  \acro{IEEE}     {Institute of Electrical and Electronics Engineers}
  \acro{IMS}      {IP Multimedia Subsystem}
  \acro{IP}       {Internet Protocol}
  \acro{LAN}      {Local Area Network}
  \acro{LENA}     {LTE-EPC Network simulAtor}
  \acro{LTE}      {Long Term Evolution}
  \acro{MAC}      {Medium Access Control}
  \acro{MBR}      {Maximum Bit Rate}
  \acro{MME}      {Mobility Management Entity}
  \acro{MPEG}     {Moving Picture Experts Group}
  \acro{MPLS}     {MultiProtocol Label Switching}
  \acro{NAS}      {Non-Access Stratum}
  \acro{NFV}      {Network Function Virtualization}
  \acro{ns-3}     {Network Simulator~3}
  \acro{ONF}      {Open Networking Foundation}
  \acro{OXM}      {OpenFlow eXtensible Match}
  \acro{P-GW}     {PDN GateWay}
  \acro{P2P}      {Peer-to-Peer}
  \acro{PCC}      {Policy Control and Charging}
  \acro{PCRF}     {Policy Control and Charging Rules Function}
  \acro{PDCP}     {Packet Data Convergence Protocol}
  \acro{PDN}      {Packet Data Network}
  \acro{QCI}      {QoS Class Identifier}
  \acro{QoE}      {Quality of Experience}
  \acro{QoS}      {Quality of Service}
  \acro{RAN}      {Radio Access Network}
  \acro{RLC}      {Radio Link Control}
  \acro{RNG}      {Random Number Generator}
  \acro{RRC}      {Radio Resource Control}
  \acro{S-GW}     {Serving GateWay}
  \acro{SAE}      {System Architecture Evolution}
  \acro{SCTP}     {Stream Control Transmission Protocol}
  \acro{SDMA}     {Semi-Distributed Mobility Anchoring}
  \acro{SDMN}     {Software Defined Mobile Networking}
  \acro{SDN}      {Software Defined Networking}
  \acro{SDWN}     {Software Defined Wireless Networking}
  \acro{SON}      {Self-Organizing Network}
  \acro{SRS}      {Sounding Reference Signal}
  \acro{TCP}      {Transmission Control Protocol}
  \acro{TEID}     {Tunnelling End ID}
  \acro{TFT}      {Traffic Flow Template}
  \acro{TLS}      {Transport Layer Security}
  \acro{TLV}      {Type-Length-Value}
  \acro{UDP}      {User Datagram Protocol}
  \acro{UE}       {User Equipment}
  \acro{VLAN}     {Virtual LAN}
  \acro{VoIP}     {Voice over IP}
  \acro{WLAN}     {Wireless Local Area Network}
\end{acronym}



\newpage
\pagenumbering{arabic}
\setcounter{page}{1}
\onehalfspacing

\section{Introdução}
\label{ch:introduction}

%******* Introduction of the Dash technology and Cloud/Fog networks****

\vspace{0.5cm}

% ----------------------------------------------------------------------------
% What is the current context in mobile network in terms of traffic and users? 
% ----------------------------------------------------------------------------
Os serviços de streaming de vídeo representam a maior parte do tráfego da Internet. De acordo com as previsões da Cisco~\cite{cisco:forecast},
%~\footnote{Visual Networking Index: atualização global de previsão de tráfego de dados móveis. Link:~\url{http://shorturl.at/hjAZ1}. Acesso em: 29 de julho de 2019.}, 
em 2021, 70\% de todo o tráfego da Internet será dominado por streaming de vídeo. Isso inclui os serviços de vídeo atuais, bem como serviços inovadores de jogos na nuvem e futuros consoles~(por exemplo, Google Stadia), enquanto que para dispositivos móveis, essa estimativa representa 78\% de todo o tráfego de dados. Essa tendência impõe novos desafios no fornecimento de vídeos com a melhor Qualidade de Experiencia~(QoE), originalmente projetada considerando o modelo de \textit{best-effort} para transmissão de dados.

% ----------------------------------------------------------------------------
% What carries have been doing to address the increasing traffic?
% ----------------------------------------------------------------------------



Os serviços de streaming de vídeo têm requisitos rigorosos, como um canal de comunicação de boa qualidade, bem como um fluxo constante e ininterrupto de informações~\cite{Immich2018WinNet}.
% A Fig. 1 Mostra ao longo de cada ano o crescimento no consumo de banda, estes tipos de cenários impõem significantes desafios para a distribuição de vídeo sobre as futuras gerações da Internet.
Para acomadar esta demanda, bem como manter o QoE do usuário, grandes players como Microsoft, Apple, Adobe e Netflix adotam o paradigma de Streaming Adaptável sobre HTTP~(HAS)~\cite{company:dashs}. Como a maioria das soluções HAS usa a mesma arquitetura, o Grupo de Especialistas em Cinema~(MPEG) propôs um padrão chamado Streaming Adaptável Dinâmico sobre HTTP~(DASH), no qual o reprodutor de vídeo pode escolher dinamicamente a taxa de bits de acordo com a largura de banda disponível percebida.
%Atualmente, esses conteudos multimidia utilizam serviços de Streaming Adaptativo Dinâmico sobre HTTP~(HAS), este paradigma trata o video multimídia como qualquer outro conteúdo comum da Web e o entrega em pequenos pedaços atraves do protocolo HTTP.
%Os serviços de transmissão de vídeo têem seu bitrate adaptados dinâmicamente.% As soluções HAS utilizam os protocolos HTTP na aplicação e TCP na camada de transporte como ilustrados na figura 1b. 
%em Streaming Adaptativo Dinâmico sobre HTTP~(DASH) são amplamente adotado por provedores de vídeo como Google, Netflix, Akamai HD e outros, nos quais o player de vídeo do cliente pode escolher dinamicamente o nível de taxa de bits de acordo com a largura de banda disponível percebida.
Esta solução emprega adaptação dinâmica em relação a variação das condições de rede para fornecer uma experiência de streaming sem interrupções~(ou pelo menos mais suave). Depois que um arquivo de mídia~(ou fluxo) está pronto a partir de um servidor fonte, ele pode ser transmitido em segmentos por um servidor HTTP padrão.


% Em cenários para as futuras gerações da Internet, tecnologias de redes sem fio, como, redes 5G, mobile-edge computing~(MEC), redes definida por software~(SDN).
%Esses fatores demonstram como o futuro da demanda da Internet pode se tornar muito grande para os ISPs e os avanços tecnológicos nas telecomunicações nas quais eles dependem. 
Atualmente, os tradicionais serviços de Streaming de vídeo são projetados para distribuir conteúdo multimídia através de grandes data centers na nuvem. Esses sistemas na nuvem geralmente usam um conjunto de servidores em que o tráfego passa pelo núcleo da rede. Além disso, de modo geral, a conexão dos dispositivos é feita por usuários estáticos e por links de Internet estáveis~\cite{sitaraman:ACD2014}. Dessa forma, a criação desses sistemas resolve parcialmente os problemas de escalabilidade, disponibilidade e interoperabilidade, mas ao mesmo tempo, apresenta novos desafios~(por exemplo, maior latência e congestionamento da rede principal)~\cite{tran:wons17,ye:ITC17,taleb:JSAC18}. Vários trabalhos na literatura destacam a computação de nevoa/borda para lidar com as novas demandas de tráfego de video que estão surgindo. Onde datacenters com menos capacidade de processamento e armazenamento conseguem prover serviços/virtualização mais próximos ao usuário final. Assim, a borda da rede pode fornecer taxas de latência que a nuvem não conseguem alcançar de outra forma~\cite{gamaUCC2019, rosarioSENSORS2018}.
% ou caches especificos alocados pelo proprio administrador da rede

%, levando em consideração as novas tecnologias que estão surgindo, como, redes 5G, mobile-edge computing~(MEC), redes definida por software~(SDN)

%HAS uses HTTP as the application and TCP as the transport-layer protocol as illustrated in Figures 1.2b and 1.3, and clients pull the data from an HTTP server. HAS solutions employ dynamic adaptation with respect to varying network conditions to provide a seamless (or at least smoother) streaming experience. Once a media file (or stream)is ready from a source, it is prepared for streaming before it is published to a standard, off-the-shelf HTTP server. The original file/stream is partitioned into segments (also calledchunks) of equi-length playback time, and multiple versions (also called representations)of each segment are generated that vary in bitrate/resolution/quality using an encoder ora transcoder (i.e., H.264, H.265, etc.). Moreover, the server generates an index file, whichis a manifest that lists the available representations including HTTP uniform resourcelocators (URLs) to identify the segments along with their availability times. During atypical HAS session, the client first receives the manifest that contains the metadata forthe video, audio, subtitles, etc., and then constantly measures certain parameters such as the available network bandwidth, buffer status, and battery and CPU levels. According tothese parameters, the HAS client repeatedly fetches the most suitable next segment amongthe available representations from the server. Table 1.1 compares the main characteristicsof the traditional streaming and HAS systems.

% ----------------------------------------------------------------------------
% What carries have been doing to address the increasing traffic?
% ----------------------------------------------------------------------------
%Em cenários para as futuras gerações da Internet, tecnologias de redes sem fio, como, redes 5G, mobile-edge computing~(MEC), redes definida por software~(SDN).
%Esses fatores demonstram como o futuro da demanda da Internet pode se tornar muito grande para os ISPs e os avanços tecnológicos nas telecomunicações nas quais eles dependem. 
%Atualmente, tradicionais serviços de Streaming de vídeo sob Demanda~(VoD) são projetados para distribuir conteúdo multimídia por grandes data centers na nuvem. Esses sistemas geralmente usam um conjunto de servidores em que o tráfego passa pel núcleo da rede; além disso, a conexão dos dispositivos é feita por usuários estáticos e por links da Internet estáveis [?]. Dessa forma, a criação desses sistemas no nível da nuvem resolve parcialmente os problemas de escalabilidade, disponibilidade e interoperabilidade, mas ao mesmo tempo, apresenta novos desafios (por exemplo, maior latência e congestionamento da rede principal) [?]. Vários trabalhos na literatura destacam a computação de nevoa/borda para lidar com as novas demandas de tráfego que estão surgindo. Onde datacenters com menos capacidade de processamento e armazenamento conseguem prover serviços/virtualização mais próximos ao usuário final. Assim, a borda da rede pode fornecer taxas de latência que a nuvem não conseguem alcançar de outra forma [?], [?].

%These two factors demonstrate how the future of Internet demand may become toooverwhelming for both ISPs and technological advancements in telecoms they rely on. Currently, the traditional VoD services are designed to distribute multimedia content by large data centers at the cloud. These systems usually use a set of servers where the traffic passes through the core network, beyond that, the devices connection are made by static end-users and Internet links stable [?]. This way, the inception of these systems at cloud-level partially solves the scalability, availability, and interoperability issues, but at the same time, introduces new ones (e.g., higher latency and core network congestion) [?]. Several works in the literature have highlighted fog/edge computing to deal with new traffic demands that are emerging. Where datacenters with less processing and storage capacity can provision virtualization/services closer to the end-user. Thus, the network edge can provide latency rates that the cloud is not able to reach otherwise [?], [?].


% ----------------------------------------------------------------------------
% What are the problems with existing networks?
% ----------------------------------------------------------------------------

%Para melhorar os serviços de vídeo, é de suma importância distribuir adequadamente os fluxos de vídeo de acordo com seus requisitos: uma infraestrutura de jogos em nuvem é um serviço interativo que precisa de atrasos reduzidos~(alguns milissegundos), enquanto uma entrega de VoD não interativa pode tolerar maior atrasos sem prejudicar a qualidade da experiência. Um gerenciamento e orquestração adequados da entrega de vídeo pela Internet é essencial para a coexistência suave de serviços de vídeo heterogêneos.

%Currently, the traditional VoD services are designed to distribute multimedia content by large data centers at the cloud. These systems usually use a set of servers where the traffic passes through the core network, beyond that, the devices connection are made by static end-users and Internet links stable [?]. This way, the inception of these systems at cloud-level partially solves the scalability, availability, and interoperability issues, but at the same time, introduces new ones (e.g., higher latency and core network congestion) [?].

%Esse novo paradigma, ao qual nos referimos como HTTP Adaptive Streaming (HAS), tratava o conteúdo da mídia como qualquer outro conteúdo comum da Web e o entregava em pequenos pedaços pelo protocolo HTTP. O HAS se tornou rapidamente a abordagem dominante para o streaming de vídeo devido à sua adoção pelos principais provedores de serviços e conteúdo. A entrega de vídeo pela Internet pública também é conhecida como streaming de vídeo over-the-top (OTT), pois o conteúdo ou o provedor de serviços de streaming geralmente é diferente do provedor de rede. O surgimento do HAS e de novos dispositivos móveis principalmente para usuários finais, com altos recursos de processamento e renderização, teve um papel fundamental no crescimento do tráfego de streaming de vídeo

%There are several works that try to solve a number of
%issues regarding the combined use of Cloud, Fog, and Edge
%computing. Generally, the existing works are based on ar-
%chitecture design and deployment issues [7]–[10]. They try
%to improve virtual machine migration, mobility adversities or
%provide smart caching for a specific type of content as well
%as to enhance video delivery with adaptive streaming and Fog
%nodes. However, the solutions presented tend to be general and
%do not take into consideration video-specific needs.

Embora muitos trabalhos de pesquisa abordem serviços de video streaming em conjunto com a computação em nuvem/névoa, existem aspectos pouco abordados em soluções atuais. Geralmente, as arquiteturas de streaming vídeo buscam diminuir a carga de tráfego e melhorar o QoE da entrega do video.
%Eles tentam melhorar a migração de máquinas virtuais, adversidades de mobilidade ou fornecer armazenamento em cache inteligente para um tipo específico de conteúdo, além de aprimorar a entrega de vídeo com streaming adaptável e nós Fog. 
Tais soluções não levam em consideração o comportamento do reprodutor de video usado pelo usuário, bem como aspectos relacionados a mobilidade do usuário nos mecanismos de tomada de decisão em ambientes multiníveis. 

%na borda da rede podem afetar aspectos relacionado ao comportamento do usuário final pouco abordados, como area de cobertura do nó, modulidade.

%mas surgem mais problemas nos cenários de Smart City: mobilidade do usuário, esquemas de cache colaborativo em várias arestas, quantidade de usuários durante multidões de flash e requisitos de streaming interativo não são totalmente considerados.
%usando ferramentas de código aberto disponíveis no mercado e sequências de vídeo reais. O principal objetivo é provar que é possível criar um ambiente multicamada real para melhorar a qualidade da entrega de vídeo. 

%Taking into account the aforementioned scenarios and new
%ones yet to be revealed in the next few years there is a growing
%need for a multi-tier video delivery architecture. It could take
%advantage, at the same time, of the elastic resource pool that
%Cloud computing provides in association with the low-latency
%and high-throughput offered by the Edge computing. This
%paper aims to advance the idea of multi-tier video delivery
%using off-the-shelf open-source tools and real video sequences.
%The main goal is to prove that it is possible to build a real
%multi-tier environment to improve video delivery quality.

%, levando em consideração as novas tecnologias que estão surgindo, como, redes 5G, mobile-edge computing~(MEC), redes definida por software~(SDN).
%O paradigma SDN pode realmente ser usado para expandir as redes existentes, afastando a dependência do fornecedor e mantendo o desempenho alcançado pelo hardware dedicado? Como as soluções propostas são avaliadas e qual é a adoção efetiva dessas soluções? Como gerenciar futuras redes 5G heterogêneas, considerando diferentes tecnologias de rádio e a explosão de conexões? Como simplificar efetivamente o gerenciamento da mobilidade nas arquiteturas atuais, explorando a visão de rede centralizada da SDN?

% ----------------------------------------------------------------------------
% Which technologies can be used to improve future networks?
% ----------------------------------------------------------------------------

% ----------------------------------------------------------------------------
% How the SDN parading is contributing to current networks?
% ----------------------------------------------------------------------------

% ----------------------------------------------------------------------------
% Why and how are we contributing in this topic?
% ----------------------------------------------------------------------------

%In an attempt to answer above questions, this ongoing doctoral research project discusses how the \ac{SDN} paradigm and the OpenFlow protocol can be integrated to the existing 4G \ac{LTE} networks to provide new solutions for some of the aforementioned problems. As contributions already developed, this project first ...


Com o objetivo de lidar com as questões acima, este projeto de pesquisa de doutorado tem como objetivo modelar mecanismos de entrega de vídeo em DASH % e de alta qualidade para ser 
a ser usado em ambientes de cidades inteligentes. Tais mecanismos propostos aproveitarão de tecnologias emergentes relacionadas à redes~(como 5G e WiFi), com o objetivo de auxiliar a tomada de decisão do serviço de streaming de video em uma arquitetura multinível. %A Figura 1 mostra, no lado esquerdo, uma arquitetura de rede de várias camadas, composta por um conjunto heterogêneo de dispositivos e aplicativos usando recursos de computação distribuídos por meio de uma tecnologia de comunicação de acesso múltiplo, como 5G e WiFi. Este projeto propõe estender o streaming de vídeo DASH para oferecer suporte à conectividade multipath simultânea [3, 4].

% ----------------------------------------------------------------------------
% How is this document organized?
% ----------------------------------------------------------------------------
Este documento está organizado da seguinte forma: \autoref{ch:background} apresenta
alguns conceitos básicos de Video Streaming Adaptativo e arquiteturas de multiníveis na Névoa/Nuvem. Em seguida, \autoref{ch:related-work} fornece uma visão geral dos trabalhos relacionados. Focando em trabalhos de Video Streaming em redes multiníveis utilizando soluções em DASH para melhorar o QoE dos usuários da rede, comparando alguns trabalhos relacionados, destacando e comparando os principais desafios neste tópico. \autoref{ch:developed} traz o trabalho realizado até o momento, o que ajudará no desenvolvimento das contribuições propostas \autoref{ch:proposal}. Finalmente, \autoref{ch:remarks} fecha o documento com algumas considerações finais.
 
%This document is organized as follows: \autoref{ch:background} presents some
%background concepts on \ac{SDN} and \ac{LTE} architectures. Then,
%\autoref{ch:integration} gives an overview of several proposals for \ac{SDMN}.
%It focuses on the use of \ac{SDN} solutions in the backhaul and core network,
%comparing some related works and highlighting major open challenges in the
%topic. \autoref{ch:developed} brings the work carried out so far, which will
%assist the development of the proposed contributions that are detailed in
%\autoref{ch:proposal}. Finally, \autoref{ch:remarks} closes the document with
%some final remarks.


\clearpage
\section{Background and concepts}
\label{ch:background}

Esta seção descreve brevemente conceitos e background tais como o paradigma Névoa/Nuvem multinível e Serviço de Video Streaming.
%This section briefly introduces background and concepts such as the \acl{SDN} paradigm and the 4G \acl{LTE} networks.

%=============================================================================%
\subsection{Computação em Névoa}

A computação em névoa tem como principal objetivo preencher uma lacuna entre a nuvem e os dispositivos finais. Existem várias definições e terminologias na literatura que abordam este paradigma na literatura. Nesta proposta, vamos adotar a definição, o qual é adotado pelo OpenFogConsortium~(OpenFog), publicada pelo Instituto Nacional da Padrões e Tecnologia%\textit{National Institute of Standards and Technology}
~(NIST)~\cite{NIST2018-FogComputingConceptualModel}: 

\begin{displayquote}

"\textit{A computação em névoa é um modelo em camadas para permitir o acesso onipresente a um contínuo compartilhado de recursos de computação escalonáveis. O modelo facilita a implantação de aplicativos e serviços distribuídos com reconhecimento de latência e consiste em nós de névoa (físicos ou virtuais), que residem entre dispositivos finais inteligentes e serviços centralizados (na nuvem)}."

\end{displayquote}
% https://nvlpubs.nist.gov/nistpubs/SpecialPublications/NIST.SP.500-325.pdf. March 2018
% https://www.openfogconsortium.org

Os nós podem ser organizados em arquiteturas multiníveis - na vertical (para dar suporte ao isolamento), na horizontal (para suportar a federação) ou pela latência entre os nós da névoa e os usuários finais. A computação em Névoa minimiza o tempo de resposta das aplicações suportadas e fornece, para os dispositivos finais, recursos de computação local e, quando necessário, conectividade de rede para serviços centralizados. %Os nós da \textit{fog} são recursos locais que podem ser organizados em arquiteturas multi-níveis e podem ser quaisquer dispositivos que oferece conexão de rede com fio/sem fio com computação, armazenamento e conectividade de rede, como switches, roteadores, smart phones, tablets, laptops, etc. 

%---------------------------------------------------------------------
\subsection{Arquitetura hierárquica multiníveis em Nuvem-Névoa}

Uma arquitetura em névoa verticalmente com N níveis tem como proposito: lidar eficientemente com a quantidade de dados que precisa ser processada e extrair dados significativos para criar mais inteligência em cada nível. Além disso, o número de camadas afeta diretamente o suporte de QoE.
%to deal efficiently with the amount of data that needs to be processed and to extract meaningful data to create more intelligence at each level. Moreover, the number of tiers impacts direclty in the QoE support.
%O OpenFog está assumindo a liderança na padronização de névoa e definiu uma arquitetura de referência que consiste em N camadas de nós, como mostra a Figura~\ref{fig:arch-multi-lvl}

Primeiramente, vamos apresentar uma arquitetura de rede com multiníveis, detalhada na Figura~\ref{fig:arch-multi-lvl}. O nível superior é composto por servidores em nuvem que podem ser localizados em um setor público ou privado. As nuvens podem ser qualquer provedor do streaming de video, por exemplo, Netflix, Amazon ou Youtube, bem como, um usuário local. % Estas nuvem operam o conteudo multimidia original sobre redes WANs.
Os 3 níveis seguintes representam a rede névoa/nuvem organizada hierarquicamente. Este  trabalho leva em consideração os nós da borda com serviços de armazenamento e computação, bem como, é ranqueado de acordo com a área de cobertura de comunicação. Nesse ecossistema multinível, o \textit{Core Network Regional Edge} pode gerenciar a coordenação em toda a cidade, por exemplo, unidade de banda de base~(BBU) ou provedor de serviços de Internet~(ISP). Seguido pela \textit{Access Network Edge}, que suporta algumas dezenas a talvez algumas centenas de nós locais no meio da nevoa, por exemplo, estação base ou ponto de acesso. Os \textit{Gateways} de borda podem ser distribuídos em nós de névoa locais, por exemplo, computadores pessoais, laptops e smartphones, onde o nó retransmite o conteúdo do vídeo por meio de comunicação com ou sem fio. Esses dispositivos têm demandas de tráfego altas e similares, podendo cooperar entre si.% Essa conexão é feita através de algum tipo de rede local~(LAN). 
%É importante notar que em um ambiente multinível os servidores são organizados em uma topologia mesh em arvore.
\vspace{0.8cm}
\begin{figure}[htb]
  \centering
  \includegraphics[scale=.45]{arch-multi-lvl}
  \caption{Main components of an multi-lvl environment.}
  \label{fig:arch-multi-lvl}
\end{figure}

%Um cenário simples é que um shopping center pode implantar muitos nós de névoa em diferentes andares para fornecer Acesso Wi-Fi e entregar alguns serviços envolvidos (ou seja, navegação interior, distribuição de anúncios, coleções de feedback) para seus clientes. No entanto, no tempo de pico, as capacidades desses nós de nevoeiro não podem servir eficientemente os clientes. Enquanto isso, o provedor de névoa, aqui é o centro comercial, pode estender sua infraestrutura, pagando os recursos de computação e armazenamento terceirizados dos nós de nuvem, que pode ser máquina virtual (VM) alugada de provedores de nuvem em uma base de pay-per-use. Todos os nós de processamento distribuídos (nuvem ou névoa) são gerenciados por um corretor de recursos, que é um componente de gerenciamento de recursos e um planejador para os fluxos de trabalho enviados pelos usuários no lado do nevoa. 
%Neste caso, uma agenda de tarefas, que pode minimizar o tempo de conclusão do fluxo de trabalho, mas corresponde a uma grande quantidade de custo monetário, não é uma solução ideal para fornecedores da fog. 
%Assim, neste artigo, propomos um algoritmo de escalonamento de tarefas que pode conseguir uma boa compensação entre o tempo de execução do fluxo de trabalho e o custo pelo uso dos recursos da nuvem. Os resultados experimentais mostram o excelente desempenho do nosso método comparado com alguns outros trabalhos.	

%Com novas possibilidades sendo criadas para oferecer melhores serviços e o funcionamento da internet. Enquanto a rede se torna mais robusta, o problema se torna mais complexo e surgem novos desafios. Para adaptar um sistema CDN com ambiente de várias camadas ao nevoeiro, diferentes características devem ser estudadas, como alocação de cache, posicionamento, substituição e seleção, geralmente, tomada de decisões em tempo real. Como diferentes tamanhos de cache, sendo alocados nas camadas para armazenar um intervalo de conteúdo para provisionar uma região. Além disso, um tamanho de cache em um dispositivo AP deve ser capaz de lidar com um conjunto de diferentes partes de conteúdo, para que os usuários finais possam ter garanias de QoE. Assim, surge um nível diferente de granularidade nos dispositivos de ponto de acesso. Se a granularidade de certas áreas se tornar muito granularidade, problemas de escalabilidade começam a aparecer. Dividir o conteúdo do nevoeiro na rede pelos APs no nível certo de granularidade é um problema complexo por si só.

%-----------------------------------------------------------------------------%
%\subsection{HTTP Adaptative Streaming and MPEG-DASG standard}
\subsection{Transmissão Adaptativa de vídeo pela Internet}
\label{sec:has-dash}

A padronização de uma arquitetura HAS prove importantes benefícios sobre sistemas de streaming tradicionais. Devido ao fato de várias empresas importantes de mídia participarem de seu desenvolvimento, este novo protocolo elimina problemas técnicos na entrega e compactação do video. Em essência, o objetivo é combinar todas as tecnologias e padrões em um, tornando o suporte a streaming contínuo em todos os dispositivos. 
%Por sua vez, visa reduzir dores de cabeça técnicas e custos de transcodificação. 
Os editores de conteúdo podem gerar um único conjunto de arquivos para codificação e streaming que deve ser compatível com o maior número possível de dispositivos, do smartphone a nuvem, além de plug-ins ou HTML5. Os consumidores não precisam se preocupar se seus dispositivos conseguem reproduzir o conteúdo que desejam assistir.

O objetivo desses esquemas é garantir um alto QoE para os usuários na presença de flutuações na largura de banda devido a alguns fatores, por exemplo, controle de congestionamento da rede, intensidade do sinal, perda de pacotes e assim por diante. Embora essas flutuações sejam bastante comuns na Internet pública, elas também podem ocorrer em redes mais privadas, como redes domésticas ou mesmo redes gerenciadas, onde geralmente há controle de admissão e diferentes ferramentas de QoS são usadas. 

% Quality Improvement for HTTP Adaptive Streaming over Mobile Networks Hung Thai Le
%To deliver videos over the Internet, the MPEG-2 transport stream (M2TS) [74] and ISO Base Media File Format (MP4) [75] are popularly used [22]. In HAS, we only consider transmissions of video segments at the application layer. After obtaining the metadata file, the client issues a series of HTTP requests to download video segments, typically in chronological order, selecting a representation for each ofthem from the set ofavailable representations.
%Para entregar vídeos pela Internet, o fluxo de transporte MPEG-2 (M2TS) [74] e o formato ISO de arquivo de mídia básico (MP4) [75] são usados popularmente [22]. No HAS, consideramos apenas transmissões de segmentos de vídeo na camada de aplicação. Após obter o arquivo de metadados, o cliente emite uma série de solicitações HTTP para baixar segmentos de vídeo, geralmente em ordem cronológica, selecionando uma representação para cada um deles no conjunto de representações disponíveis.
%
%As aplicações de video streaming baseadas em MPEG-DASH, uma tecnologia que se adapta dinamicamente às mudanças nas condições da rede, solicitando conteúdo em segmentos codificados com taxas de bits diferentes e que é usada pelos principais serviços de streaming como Netflix e YouTube.
%a technology that dynamically adapts to changing network conditions by requesting content in chunks encoded at different bitrates, and which is used by major streaming services like Netflix and YouTube.
%In this proposal, we focus on video streaming applications based on MPEG-DASH (Dynamic Adaptive Streaming over HTTP), a technology that dynamically adapts to changing network conditions by requesting content in chunks encoded at different bitrates, and which is used by major streaming services like Netflix and YouTube.
%There are several key benefits in the adaption of this new standard. Due to the fact that several major media companies took part in its development, the new protocol will eliminate technical issues in delivery and compression. In essence, it aims to combine all of the technologies and standards into one, making streaming support seamless on all devices. In turn, it aims to reduce technical headaches and transcoding costs. Content publishers can generate a single set of files for encoding and streaming that should be compatible with as many devices as possible, from mobile to OTT, as well as to the desktop via plug-ins or HTML5. Consumers will not have to worry about whether their devices will be able to play the content they want to watch.

\begin{figure}[htb]
  \centering
  \includegraphics[scale=.5]{has-arch}
  \caption{Arquitetura de distribuição HAS.}
  \label{fig:has-arch}
\end{figure}
%By parsing the MPD, the DASH client learns about the program timing, media-content availability, media types, resolutions, minimum and maximum bandwidths, and the existence of various encoded alternatives of multimedia components, accessibility features and required digital rights management (DRM), media-component locations on the network, and other content characteristics. 
%Ao analisar o arquivo MPD, o cliente DASH aprende sobre o tempo do programa, disponibilidade de conteúdo de mídia, tipos de mídia, resoluções, larguras de banda mínima e máxima e a existência de várias alternativas codificadas de componentes de multimídia, localizações dos componentes de mídia na rede e outras características do conteúdo
%
%Out of Scope
%The MPEG-DASH specification only defines the MPD and the segment formats. The delivery of the MPD and the media encoding formats containing the segments, as well as the client behavior for fetching, adaptation heuristics, and playing content, are outside of MPEG-DASH’s scope.

Na Figura~\ref{fig:has-arch}, conforme ilustrado, o cliente e servidor HAS usam o protocolo HTTP na camada da aplicação para realizar todas as requisições/respostas necessárias. 

No lado do servidor, assim que um arquivo de mídia~(ou fluxo) estiver pronto, ele será preparado para transmissão antes de ser publicado em um servidor HTTP padrão. O arquivo/fluxo original é particionado em segmentos~(também chamados de \textit{chunks}) de tempo de reprodução equivalente, e são geradas várias versões (também chamadas de representações) de cada segmento que variam em taxa de bits/resolução/qualidade usando um codificador ou um transcodificador. 
Além disso, o servidor gera um arquivo manifesto chamado de descritor da apresentação na mídia~(MPD), que lista as representações disponíveis, incluindo informações como o tempo do video, disponibilidade de conteúdo, tipos de mídia~(ou seja, H .264, H.265, etc.), resoluções, larguras de banda mínima e máxima e a existência de várias alternativas codificadas de componentes de multimídia, localizações dos segmentos de mídia na rede e outras características do conteúdo.
%URLs para identificar os segmentos e seus tempos de disponibilidade. 

No lado do cliente HAS, primeiramente, o reprodutor de video solicita o manifesto ao servidor HTTP e analisa as informações citadas acima contidas no arquivo. Desta forma, ele pode começar a solicitar segmentos sequencialmente e adaptar-se às condições da rede dinamicamente usando sua lógica adaptativa de taxa de bits~(ABR). Os esquemas ABR também levam em consideração o buffer de reprodução, recursos do dispositivo, preferências do visualizador, além de recursos de conteúdo, com pesos diferentes. 
Como a QoE do espectador precisa ser determinado em tempo real durante a reprodução, geralmente, são usadas métricas objetivas, incluindo o número de interrupções, duração do atraso na inicialização, frequência e quantidade de oscilações de qualidade do vídeo. Por padrão, o HAS não exige nenhum esquema de adaptação específico, deixando aos desenvolvedores de sistemas inovar e implementar seu próprio método.


%O analisa o MPD arquivo MPD , o player do cliente extrai 
%Em seguida, as soluções HAS realizam a adaptação dinâmica em relação às estas variaveis para fornecer uma experiência de streaming contínua~(ou pelo menos mais suave). 

%Durante uma sessão HAS atípica, o cliente recebe primeiro o manifesto que contém os metadados para vídeo, áudio, legendas etc. e depois mede constantemente certos parâmetros, como largura de banda de rede disponível, status do buffer e níveis de bateria e CPU. De acordo com esses parâmetros, o cliente HAS busca repetidamente o próximo segmento mais adequado entre as representações disponíveis do servidor.
%A Tabela 1.1 compara as principais características dos sistemas tradicionais de streaming e HAS.

%-----------------------------------------------------------------------------%
%\subsubsection{Arquitetura Hierarquica para Bitrate Schemes}
%\label{subsec:bitrate-schemes}
%
%%Implementing a multi-tier architecture has a dual purpose: to deal efficiently with the amount
%%of data that needs to be processed and to extract meaningful data to create more intelligence at
%%each level. Moreover, the number of tiers impacts direclty in the QoE support.
%
%Each ABR scheme proposes many criteria for bitrate decisions, where they work only under
%indirect or implicit assumptions and specific scenarios, and focuses on a specific deployment or
%different network characteristics. Currently, there is a lack of a general consistent framework
%that can formally evaluate and compare different bitrate adaptation schemes, and test and
%verify the efficiency of their components. To the best of our knowledge, only a few algorithms
%formally describe what objective they want to optimize, and thus, it is challenging to make
%an effective comparison.
%
%In this part, we provide a feature comparison between various state-of-the-art bitrate
%adaptation schemes, and are available in terms os the following aspects:
%
%\begin{itemize}
%\item Heurisic:	
%\item Fairness:
%\item \ac{QoE}:
%\item \ac{QoE} optimization:
%\item Number of Clients:
%\item Content type:
%\item Heterogeneity:
%\item SVC support: Does the adaptation algorithm support the streaming of SVC-encoded video?
%\item BG Traffic: Does the paper include background traffic in their experimental tests?
%\end{itemize}

\subsubsection{Fatores de influencia no QoE}

O QoE é uma avaliação da satisfação do usuário com o conteúdo exibido na tela. O QoE é o grau de prazer ou aborrecimento de uma pessoa em relação a um aplicativo, serviço ou sistema. Assim, QoE se refere à percepção da pessoa sobre o conteúdo exibido no dispositivo. Nesta seção, nós discutimos alguns fatores que mais influenciam o QoE do usuário. 

%Juntamente com todas as possibilidades trazidas pela computação em nevoa em multiníveis, há uma série de desafios que impedem sua plena realização. 

% Quality Improvement for HTTP Adaptive Streaming over Mobile Networks Hung Thai Le
\begin{itemize}

\item \textit{Atraso inicial:} Toda transmissão de video precisa de uma certa quantidade de dados iniciais antes que a decodificação e reprodução possar ser inciada, essa transferência de dados antes da reprodução do video é chamada de atraso inicial. Está fase também é conhecida como bufferização inicial. Durante esse atraso incial o cliente não precisa preencher seu buffer com uma grande quantidade de dados, isso pode causa um longo atraso inicial. No entanto, um nível maior de buffer ajuda o cliente a evitar eficientemente os fluxos insuficientes no buffer.

\item \textit{Interrupções:} É o congelamento da reprodução do video devido ao esvaziamento do buffer. Especificamente, um fluxo insuficiente de buffer é seguido por um período de buffer, onde o cliente precisa armazenar em buffer rapidamente uma quantidade de dados de vídeo para retomar a reprodução. 
Em alguns serviços de streaming, um rebuffering pode ser tratado de forma semelhante a um buffer inicial. No entanto, os impactos dos dois tipos de tempo de espera são diferentes para os usuários. Ao contrário do atraso inicial, que é o tempo de espera antes do serviço e é bem conhecido, a interrupção ocorre inesperadamente no serviço e, portanto, a percepção para o usuário é muito pior. Em [88], os autores mostram que existe uma relação exponencial entre o número de interrupções e a qualidade do vídeo. Embora achem que os usuários podem tolerar no máximo uma interrupção de alguns segundos durante uma sessão.

\item \textit{Qualidade perceptual:}
Uma taxa de bits de vídeo mais alta geralmente fornece uma qualidade de vídeo melhor. Em [89] observa-se um forte efeito da qualidade recente do vídeo, ou seja, uma qualidade mais alta no final de uma sessão resulta em maior experiência do usuário.
Em [90], verificou-se que, além da taxa de bits média do vídeo, o tempo em cada camada individual tem um impacto significativo na qualidade do vídeo.
A variação da qualidade em uma sessão é determinada pela amplitude de comutação e pela frequência de comutação. Vale ressaltar que cada comutação é representada por uma alteração de qualidade diferente de zero (por exemplo, taxa de bits de vídeo, versão de vídeo etc.) dos dois segmentos de vídeo conservadores. Uma amplitude de chaveamento indica o grau de uma mudança na qualidade, enquanto a frequência de chaveamento pode ser representada por vários interruptores em toda a sessão
%\item \textit{Amplitude qualidade:}
%\item \textit{troca de qualidade:}

\item \textit{Latencia ao vivo:} No caso de transmissão ao vivo, um fator de influência de qualidade adicional que desempenha um papel importante é a latência ao vivo (ou o atraso da captura para exibição). Como discutido em [98], os principais componentes de atraso incluem preparação de conteúdo, atraso de segmentação, busca assíncrona de segmentos de mídia, tempo de download HTTP, tempo de buffer no buffer e tempo de decodificação. Entre esses componentes, o tempo de buffer no buffer, que depende do nível atual do buffer, contribui com uma parte significativa para a latência ao vivo, especialmente no streaming sob demanda, onde o tamanho do buffer é de dezenas de segundos.

\end{itemize}

% No streaming clássico baseado em HTTP (ou seja, streaming progressivo de download), os principais fatores de influência na qualidade do vídeo são atraso inicial, interrupção e amplitude de qualidade [88, 91]. No HAS, o cliente altera a qualidade do vídeo entregue durante uma sessão, o que introduz qualidade
%variação como um fator de influência adicional na qualidade perceptiva [92, 93]. Em [93], verificou-se que os parâmetros relacionados à estratégia de adaptação de vídeo (ou seja, comutadores de representação) devem ser considerados em uma escala de tempo maior (até alguns minutos) e que são mais importantes que os parâmetros relacionados à codificação de vídeo (por exemplo, resolução, taxa de quadros, parâmetro de quantização, etc.), que influenciam apenas na ordem de alguns segundos
\section{Trabalhos Relacionados}
\label{ch:related-work}

Está seção busca mostrar os principais trabalhos relacionados a soluções em sistemas CDN e as tecnologias usadas para transmissão de conteúdo multimidia. 
Os projetos de arquitetura estão diretamente relacionados ao paradigma da computação em névoa para aplicativos de baixa latência. A disseminação de conteúdo concentra-se em reduzir a redundância da transmissão de dados nos nós de borda.
%The architecture designs directly related to fog computing paradigm for low latency applications. Content dissemination focus on reduce redundancy of data transmission on edge nodes.


% CDN-as-a-Service Provision Over a Telecom Operator’s Cloud
% Managing QoS Constraints in a P2P-Cloud Video on Demand System.
% OpenCache: A Software-defined Content Caching Platform.

\subsection{Arquiteturas Nuvem-Névoa}
\label{subsec:arch-cloud-fog}
% [ICC'15] Joint Content-Resource Allocation in Software Defined Virtual CDNs
% [CLCN'17] Optimal and Cost Efficient Algorithm for Virtual CDN Orchestration
% [CLCN'16] Scalable and Cost Efficient Algorithms for Virtual CDN Migration
% [ComNet'17] OPAC: An optimal placement algorithm for virtual CDN

Hatem \textit{et al.} [1][2][3][4] destaca os princípios de Rede Definida por Software~(SDN) e Virtualização de Função de Rede~(NFV). A abordagem baseada em SDN/NFV permite funções específicas de virtualização em servidores remotos. Dessa forma, as migrações de serviços de conteudo multimidia podem ser virtualizadas em diferentes \textit{datacenters}. Os problemas de orquestramento e cache são abordados, seu trabalho desenvolve um algoritmo exato para decidir os locais ideais para alocação do serviço. O algoritmo proposto, incluindo o cache de conteúdo e o redirecionamento de solicitações, é introduzido com algumas restrições de QoE, de rede e sistema operacional. 
Desta forma, o gerenciamento do CDN é feito de maneira centralizada e utiliza o usuário final como alvo para fazer a comunicação Dispositivo para dispositivo, mas não explora a mobilidade. As solicitações de usuários finais serão redirecionadas para um local de nuvem de borda ideal, sem dispositivos de borda de nível com diferentes lantências. 

Llorca \textit {et al.} [ICC'15] propõem uma rede de cache virtual implementada totalmente em software através de uma infraestrutura de rede em nuvem distribuída programável que pode ser consumida e otimizada elasticamente usando informações globais sobre condições de rede e requisitos de serviço chamados SDvCDN. Esta abordagem aborda os problemas de posicionamento (localização da instalação), roteamento~(rede de fluxo) e alocação de recursos~(design de rede).

% [SENSORS'18] Service Migration from Cloud to Multi-tier Fog Nodes for Multimedia Dissemination with QoE Support
Rosario \textit{et al.} [1] apresenta uma arquitetura para servicos de migração ao vivo de VM da nuvem para multiniveis da fog. O cenario experimental a nuvem distribui o conteudo de video para os diferentes niveis da fog. A arquitetura é baseada no paradigma sdn para, distribuição de video com suporte a QoE. 
The work split the multi-tier fog in three tier in order to their cover, storage, upload and download capacity. Important aspects could be tailored to support generic content and IoT environments, besides work with both private and public clouds. A divisão da nuvem em multiniveis se dá pelas caracteristicas do  dispositivos conectado a nuvem, e não por qualquer interconexão entre esses aparelhos. The paper tem como focus prover tecnologias capazes de tornar este ambiente factivel, e melhorar o provisionamento de conteudo de servicos de stream de video.


% [ICC'17] Content Delivery Network Slicing: QoE and Cost Awareness
Retal \textit{et al.} [ICC'17] propõe uma plataforma de \textit{CDN as a Service (CDNaaS)} onde os usuário podem criar um \textit{slice} de CDN incluindo cache, transcodificador e \textit{streamers}, em ordem de gerenciar uma quantidade de videos para seus usuários. (Aborda CDN na nuvem)
% [JSAC'18] Optimal VNFs placement in CDN Slicing over Multi-Cloud Environment
Benkacem \textit{et al.} [JSAC'18] introduzir uma plataforma CDNaaS na qual um usuário pode criar uma fatia CDN definida como um conjunto de rede distribuída isolada de servidores de borda em domínios com várias nuvens, em que um servidor de borda hospeda um único VNF, como cache virtual, transcodificador virtual, streamer virtual e CDN- coordenador específico de fatia para o gerenciamento do ciclo de vida dos recursos de fatia e também para gerenciar vídeos e assinantes enviados. Essa plataforma foi projetada para ter o nível máximo de flexibilidade para reduzir uma fatia de CDN no topo de diferentes IaaS (Infraestrutura como Serviço) públicas e privadas, como Amazon AWS service, Microsoft Azure, Rackspace e nuvem gerenciada OpenStack. Além disso, a plataforma emprega mecanismos e algoritmos que criam fatias de CDN com reconhecimento de QoE com boa relação custo-benefício, envolvendo uma colocação ideal levando em conta o nível de QoE desejado. Portanto, o objetivo deste trabalho é encontrar um custo eficiente de CDN, respeitando, por um lado, os requisitos do proprietário da CDN em termos de QoE e, por outro lado, a infraestrutura em nuvem e seu custo.

% Adaptive Video Streaming with Network Coding Enabled Named Data Networking
Saltarin \textit{et al.} propõe uma arquitetura adaptável de streaming de vídeo pela NDN que usa codificação de rede para permitir o streaming ideal de vídeo com vários caminhos. Assim, o uso de vários caminhos para conectar os clientes às fontes aumenta a largura de banda vista pelos clientes, permitindo que os mecanismos de adaptação de qualidade do DASH convergam para melhores qualidades de vídeo do que com o uso de um único caminho de comunicação. Os clientes podem transmitir interesses por todas as suas interfaces de rede (por exemplo, LTE e Wi-Fi) para recuperar os pacotes de dados que compõem o conteúdo solicitado.
%Fig. 1. Devices retrieving Data packets over LTE and Wi-Fi: (a) multi-source unicast; (b) single-source multicast; (c) multi-source multicast (butterfly network).

%Shen \textit{et al.} [6] works with a set of cache proxy services to analyze the cache miss occurrences. This work implements a reactive approach where cache proxies download the chunks of multimedia content when requested.
Shen \textit{et al.} trabalha com um conjunto de serviçosde cache, afim de analisar as ocorrências de falta de cache em servidores proxy. Este trabalho implementa uma abordagem reativa na qual os proxies de cache baixam os blocos de conteúdo multimídia apenas quando solicitados. Utilizando teoria da probabilidade para deduzir os valores mais adequados dos parâmetros críticos e fornecer orientações significativas para a seleção de valores para melhorar o QoE dos usuários.

\subsection{Serviços de Video Streaming Adaptativos}

% [1] QoE-fair Resource Allocation for DASH Video Delivery Systems
Cicco~\textit{et al.}~[1] aborda implementa uma estratégia de alocação de recursos justa. Para melhorar o QoE dos usuários, técnicas de engenharia de tráfego baseadas em slincing rede foram utilizadas. Ele mostra que a estrutura de otimização do Problema do Fluxo de Multi-Commodities~(MCFP) pode ser uma metodologia adequada para impor justiça em relação ao QoE dos usuários. Este artigo, em particular, primeiro mostra como converter nosso problema de alocação justa de recursos de QoE para um MCFP e, em seguida, propõe uma abordagem de agrupamento de tráfego para reduzir sensivelmente o número de fatias de rede e tornar o problema resultante tratável para distribuição de vídeo plataformas que atendem a um grande público. Essa abordagem de agrupamento atribui sessões de vídeo com base em uma métrica de similaridade proposta que depende da qualidade do vídeo.

% Want to Play DASH? A Game Theoretic Approach for Adaptive Streaming over HTTP
Bentaleb~\textit{et al.} desenvolve um Algoritmo de Teoria dos Jogos, um novo esquema de ABR orientado ao cliente que se esforça para selecionar a melhor taxa de bits baseada na moderna teoria dos jogos (GT) [13, 25]. Nossa solução permite uma colaboração eficiente entre diferentes entidades do DASH de maneira distribuída, sem sobrecarga explícita de comunicação, respeitando os requisitos de decisão dos players existentes do DASH e considerando o tráfego cruzado e as diferentes condições da rede. O GTA tem como objetivo alcançar um QoE de visualizador alto e estável.


% Client-Server Cooperative and Fair DASH Video Streaming
Altamimi \textit{et al.} 

%Google proposal
Zhang \textit{et al.} [5] concentra-se no lado do usuário, executando o nível médio de taxa de bits pelo algoritmo de adaptação à taxa de bits e a influência da variação do tamanho de segmentos para melhorar a QoE, enquanto que 

Poliakov~\textit{et al.} [3] implanta um streaming de vídeo DASH com várias fontes. O player do DASH-client pode baixar multiplos segmentos, ao mesmo tempo, através de diferentes conexões na nuvem. 

Archer \textit{et al.} [8] propõe um algoritmo para lidar com as réplicas de cache para provisionamento de vídeo com largura de banda flash, o que é um gargalo crítico.

\subsubsection{Comparação entre trabalhos relacionados}
\label{subsec:applications}

As abordagens mencionadas em \autoref{subsec:arch-cloud-fog} podem diminuir a carga de tráfego 
e melhorar a QoE. No entanto, também existem armadilhas devido ao comportamento egoísta totalmente isolado (ou seja, essas soluções estão funcionando independentemente, sem coordenação) dos players do HAS.

Os trabalhos de streaming de video na seção 3.2 consegue este tipo de problema, surgem problemas nos cenários da Cidade Inteligente: mobilidade do usuário, esquemas de cache colaborativo em multiplos níveis, quantidade de usuários durante multidões de flash não são totalmente considerados. Neste projeto, pretendemos projetar um sistema de entrega de vídeo que considere esses problemas para melhorar a qualidade da experiência para uma variedade de necessidades de streaming de vídeo, incluindo requisitos de baixa latência.

%The aforementioned approaches could decrease the traffic load and improve QoE, but more
%issues arise in Smart City scenarios: user mobility, collaborative cache schemes over multi-edge,
%the amount of users during flash crowds, and interactive streaming requirements are not fully
%considered. In this project we aim to design a video delivery system that considers such issues
%to improve quality of experience for a range of video streaming needs, including low latency
%requirements.

Nesta parte, fornecemos uma comparação entre os trabalhos discutidos acima.
%de recursos entre vários esquemas de adaptação de taxa de bits de ponta em cada categoria. 
A Tabela~\ref{tab:comparison} resume essa comparação para cada artigo pesquisado em termos dos seguintes aspectos:

\begin{itemize}

\item Escalabilidade: O trabalho leva em consideração a escalabilidade apresentada em ambientes de cidades inteligentes? O número de usuários pode variar de acordo com o tempo e a mobilidade dos usuários.

\item \# de usuários: Quantos clientes estão incluídos nos experimentos? Único indica apenas um cliente, vários (poucos) indicam menos de 10 clientes e múltiplos (muitos) indicam mais de 10 clientes.

\item Heteregeneidade: O trabalho leva em consideração dispositivos com diferentes resoluções em seus experimentos?

\item Justiça: o trabalho leva em consideração a justiça entre vários clientes que compartilham a rede? Algumas soluções compartilham igualmente a largura de banda entre os clientes, indicados pelo BW, outros compartilham a largura de banda com base na qualidade perceptiva ou no QoE, indicados por QT e QoE, respectivamente.

\end{itemize}


\begin{table}[htb]
  \caption{Comparação com trabalhos relacionados.}
  \label{tab:comparison}
  \centering
  \scriptsize
  \begin{tabular}{p{3.6cm}p{2cm}p{2cm}p{2.2cm}p{2cm}p{2cm}}
    \toprule
    \textbf{Reference} &
    \textbf{Esquemas ABR} &
    \textbf{Escalabilidade} &
    \textbf{Mobilidade} &
    \textbf{\# de \newline usuários} &
    \textbf{Aproximação \newline Cooperativa} \\
    \midrule

    Hatem~\textit{et al.}~\cite{Ahmad2013a, Ahmad2013b} &
    Flow-based & Yes & \ac{DMM} gateways & No & No \\
    \addlinespace
    \addlinespace

    Llorca~\textit{et al.}~\cite{Banerjee2013} &
    Tunnel-based & Not discussed & Not discussed & Yes & Test bed \\
    \addlinespace
	\addlinespace
    Rosáio~\textit{et al.}~\cite{Basta2013a, Basta2014} &
    Tunnel-based & Yes & Not discussed & No & No \\
    \addlinespace
	\addlinespace
    Retal~\textit{et al.} \cite{Cho2014} &
    Tunnel-based & Not discussed & Aware & Yes & Test bed \\
    \addlinespace
	\addlinespace
    Benkacem~\textit{et al.} \cite{CostaRequena2014} &
    Flow-based & Yes & Aware & No & Test bed \\
    \addlinespace
	\addlinespace
    Saltarin~textit{et al.} \cite{Ghazisaeedi2013} &
    Tunnel-based & Not discussed & Not discussed & Yes & Simulation \\
    \addlinespace
	\addlinespace
    Shen~\textit{et al.} \cite{Guerzoni2014} &
    Flow-based & Not discussed & Aware & Yes & No \\
    \addlinespace
	\addlinespace
%-------------------------------------------------------------------------------------

    Cicco~\textit{et al.} \cite{Gurusanthosh2013} &
    Tag-based & Yes & \ac{DMM} anchors & No & Analytical \\
    \addlinespace
	\addlinespace
	
    Bentaleb~\textit{et al.} \cite{Hampel2013} &
    Tunnel-based & Yes & Not discussed & Yes & No \\
    \addlinespace
	\addlinespace
	
    Altamimi~\textit{et al.} \cite{Jin2013a} &
    Tag-based & Yes & Aware & No & Test bed \\
    \addlinespace
	\addlinespace
	
    Zhang~\textit{et al.} \cite{Karimzadeh2014} &
    N/a & Yes & \ac{DMM} solutions & Not discussed & No \\
    \addlinespace
	\addlinespace
	
    Polliakov~\textit{et al.} \cite{Kempf2012a} &
    Tunnel-based & Not discussed & Not discussed & No & Test bed \\
    \addlinespace
	\addlinespace
	
    Archer~\textit{et al.} \cite{Kuklinski2014b} &
    N/a & Yes & Main focus & Not discussed & No \\
    \addlinespace
	\addlinespace
	
    \bottomrule
  \end{tabular}
\end{table}

\include{integration}
\clearpage
\section{Trabalho Desenvolvido}
\label{ch:developed}

Esta seção apresenta o trabalho realizado até o momento para auxiliar o desenvolvimento 
no projeto de pesquisa de doutorado. 

%This section presents the work carried out so far that is assisting the
%development of this doctoral research project. \autoref{sec:module} talks about
%some available software tools for performance evaluation and presents the new
%OpenFlow module for simulations. \autoref{sec:scenario} introduces the proposed
%\ac{SDN}-enabled \ac{LTE} simulation scenario, describing its design targets
%and the network topology. Finally, \autoref{sec:controller} describes the
%enhanced OpenFlow controller for the \ac{SDN}-enabled \ac{LTE} network.
% The work presented here is potential precursor of this doctoral research
% project.


%-----------------------------------------------------------------------------%
%\subsubsection{Comparison between the developed works}
%\label{subsec:applications}
%
%In this part, we provide a feature comparison between various state-of-the-art bitrate
%adaptation schemes in each category from the taxonomy in Figure 3.1. Table 3.1 summarizes
%this comparison for each surveyed paper in terms of the following aspects:

%-----------------------------------------------------------------------------%
\subsection{Formulação de Programação Linear Inteira~(ILP)}
\label{subsec:applications}

Em uma arquitetura multinível, os streamings de video tem um QoE diferente de acordo com os seus requisitos~\cite{judyLATINCOM2017}. Por exemplo, streaming em tempo real, como detecção online e streaming de armazenado fim-a-fim são sensíveis a atrasos. Portanto, esses streamings devem ser processados o mais próximo possível do usuário final, de preferência em nós localizados no primeiro nível da névoa, enquanto que conteúdos de Vídeo sob Demanda~(VoD) aceitam um atraso maior. Desta forma, vamos identificar cada streaming de video requisitado pelos usuários finais por um identificador, a partir de agora simbolizado por "c".
Um nível de servidores, indicado por $N_{c}$, é um conjunto de nós que atende o QoE necessário para fornecer serviços de um streaming de video específico, e nós definimos como um intervalo na forma de $N_{c} \in [a, b)$. onde $a_{c} < b_{c}$ e,

\vspace{0.5cm}
\begin{itemize}

\item  $c \in \{1,...,v\}$

\item  $a_{c} = \{a_{1},a_{2},...,a_{v}\}$

\item  $b_{c} = \{b_{1},b_{2},...,b_{v}\}$

\item  $a_{c},b_{c} \in N$

\end{itemize}
\vspace{0.5cm}


O balanceamento de carga em uma rede multinível Névoa/Nuvem para streaming de video pode ser formulado através de ILP.
%Um formulação de programação linear inteira pode ser modelada como um caso simples de .
Suponha que existe $n$ usuários conectados na rede, $m$ servidores e $q$ streaming de video, nós queremos encontrar um \textit{matching} apropriado entre eles. Um \textit{matching} entre um usuários, servidores e vídeos é dado por uma tripla $(i,j,k)$. Neste caso a tripla pode ser entendida como uma sessão aberta de streaming através do protocolo HTTP. O conjunto de usuários, servidores e vídeos são denotados por $U$, $N$, $V$, respectivamente.
Os vértices representam nós de rede e as arestas são os links de rede podendo ser com ou sem fio. Os links podem ser definidos por conexões físicas ou virtuais

% \begin{itemize}
% \item $\forall (i,j,k) \in S$, $i \in U$ e $j \in N$ e $k \in V$.

% \item $\forall i \in U$, exite no máximo um tripla $(i,j,k) \in S$.

% \item $\forall j \in N$, exite no máximo uma tripla $(i,j,k) \in S$.

% \item $\forall k \in V$, exite no máximo uma tripla $(i,j,k) \in S$ $\forall i \in M_{U}(i)$, onde o conjunto de download de todos os segmentos .
% \end{itemize}

%Com $V$ sendo o conjunto de todos os vertices, definimos uma atribuição $S$ como um conjunto de triplas $(i,j,k)$, considerando a arquitetura da Figura~[1]:

%\begin{itemize}
%\item $\forall (i,j,k) \in S$, $i \in D$ and $j \in U$ e $k \in V$.
%
%\item $\forall i \in D$, exite no maximo uma tripla $(i,j,k) \in S$.
%
%\item $\forall j \in U$, exite no maximo uma tripla $(i,j,k) \in S$.
%
%\item $\forall k \in V$, exite no maximo uma tripla $(i,j,k) \in S$ $\forall i \in M_{D}(i)$.
%\end{itemize}
 
O balanceamento de carga resolve uma ILP que busca o valor da variável $x_{i,j,k} \in \{0, 1\}$. $x_{i,j,k}$ é uma variável binária que assume os seguintes valores,

\vspace{0.5cm}
\begin{equation}\label{total_capacity_loss}
x_{i,j,k} =
\left\{\begin{matrix}
1, & \text{se um usuário \textit{i} requisita ao servidor \textit{j} um video \textit{k}}& \\ 
0, & \text{Caso contrário} & 
\end{matrix}\right.
\end{equation}
\vspace{0.5cm}

%Para simplificar esta formulação nós estamos considerando que os segmentos de video é requisitado a apenas um servidor fonte, e utiliza um interval de tempo discreto. Assim, $T = {1, ..., T max }$, onde $T_{max}$ é o tempo maximo que o video levaria para executar no nós mais rapido da fog. $T_{max}$ pode ser calculado como 
%
%\begin{equation}\label{minimize}
%T_{max} = \sum^{m}_{k} min(TI_{k | k \in N_{H}})
%\end{equation}

%Para melhor a latência da reprodução do video, nós devemos  minimizar a latencia de um nó para todas requisiçoes de segmentos do video. 

Para simplificar esta formulação nós estamos considerando que os segmentos de video são requisitados a apenas um servidor fonte, bem como o serviço de video streaming já está em execução em seus respectivos servidores. Colocando as limitações abaixo:% An initial approach to the QoS-aware schedule is given by the following optimization problem:


\vspace{0.5cm}

%\begin{equation}\label{minimize}
%\text{minimize} \ \ \
%\sum^{\left | D \right |}_{i=1} 
%\sum_{\{j | (i,j,k) \in A\}}
%\sum_{\{k | (i,j,k) \in A\}}
%\frac{ c_{ijk} }{ m_{i}} \ast x_{ijk}
%\end{equation}

Minimizar

\begin{equation}\label{maximize}
\sum_{i \in U} 
\sum_{j \in N_{c}}
\sum_{k \in V}
a_{ij} \ast x_{ijk}
\end{equation}

Sujeito a

\begin{equation}\label{bound_1}
%\sum_{\{j | (i,j,k) \in A\}}
\sum_{ j \in N_{c} }
\sum_{ j \in V}
x_{ijk} = 1,  \forall i \in U
\end{equation}

\begin{equation}\label{bound_1}
%\sum_{\{j | (i,j,k) \in A\}}
%\sum_{\{ | (i,j,k) \in A\}}
\sum_{ i \in U}
\sum_{ k \in V }
x_{ijk} \leq 1,  \forall j \in N_{c}
\end{equation}

\begin{equation}\label{minimize}
\sum_{i \in U}
\sum_{j \in N_{c}}
bw_{ij} x_{ij}
\leq Bw_{ij}
\end{equation}

\begin{equation}\label{minimize}
x_{ij}  \in  \{0, 1\}, \forall i \in U,j \in N_{c}
\end{equation}

\begin{equation}\label{minimize}
Bw_{ij} \geq  0, \forall i \in U,  \forall j \in N_{c}
\end{equation}
\vspace{1.2cm}

Para melhorar o desempenho da rede associamos uma latência $a_{ij}$, desta forma, nós devemos atribuir todos os usuários aos respectivos streaming de video afim de minimizar a latência total.
%A função objetivo minimiza o makepan do aplicativo fornecido por um planejamento.
%Para melhorar a latência da reprodução do video, nós devemos  minimizar a latencia de um nó para todas requisiçoes de segmentos do video. 
Logo, a função objetivo busca essa minimização. A restrição~(3) requer que cada usuário seja atribuído a exatamente um servidor. A restrição~(4) garante que cada usuário seja atribuído a no máximo um servidor. Afirma também que, se o número de vídeos dentro de um servidor é maior do que o número de sessões abertas, alguns dos vídeos permanecem sem atribuição.% A terceira restrição garante que os slots de download em uma similaridade com a classe de download diferem de segmentos de video requisitados.

A Restrição (5) estabelece as restrições de capacidade de banda disponível, que evita a atribuição de mais streaming de video do que a capacidade total disponível para cada nó de névoa.

A restrição (6) define o domínio para a variável $x_{i, j, k}$ na formulação. A restrição (7) estabelece que a variável $Bw$ deve ser um número inteiro não negativo.

%-----------------------------------------------------------------------------%
\subsection{Avalização sobre o provisionamento de Video em Diferentes Níveis}
\label{subsec:evaluation}

Alguns experimentos foram realizados para investigar como os esquemas de decisão ABR no lado do cliente se comportam. Para esta simulação foram abordados os seguintes algoritmos:~\textit{i) Rate}: O controlador ABR solicita a taxa de bits mais alta
que a rede pode suportar com base na estimativa de largura de banda disponível obtida de segmentos baixados anteriormente;~\textit{ii) Buffer}: O controlador ABR usa a ocupação do buffer de reprodução para selecionar uma taxa de bits adequada para futuros pedaços que mantêm o buffer no nível de ocupação desejado;~\textit{iii) Hybrid}: Esse tipo de controlador ABR combina as duas heurísticas mencionadas acima para decidir qual o nível de taxa de bits do próximo segmento a ser baixado.
 

%mecanismos de controle podem trabalhar juntos na rede LTE habilitada para SDN.

Para implementar os servidores DASH e os usuários que permitem o streaming de video adaptativo, utilizamos o Adaptive Multimedia Streaming~(AMuSt).
A estrutura do AMuSt fornece um conjunto de aplicativos para produzir e consumir vídeo adaptável, com base no padrão DASH~\cite{kreuzberger2016amust}. A funcionalidade do DASH é fornecida pela biblioteca libdash~\cite{mueller2013ICMEW}, uma biblioteca de código aberto que fornece uma interface para o padrão DASH. Atualmente, libdash é o software de referência oficial do padrão DASH.
Consideramos que os usuários estão interessado em um vídeo disponível com dez representações diferentes, de taxa de bits \{235kbps, 375kbps, 560kbps, 560kbps, 750kbps, 1050kbps, 1750kbps, 2350kbps, 3000kbps, 4300kbps, 5800kbps\}, que são um subconjunto usados pela Netflix~\cite{netflix:representation}.
Cada representação é dividida em segmentos de 2 segundos. Cada experimento foi executado 10 vezes com tempo do execução de 600 segundos.

%We consider that the end-users are interested in a video available in three different representations, Q = \{480p, 720p, 1080p\} with bitrates \{1750kbps, 3000kbps, 5800kbps\}, respectively, that are a subset of the ones used by Netflix in the past [11]. Each representation is divided into a set of 50 segments, each of a duration of 2 seconds.

%Primeiro, focamos em um cenário de rede de caminhos múltiplos simples (consulte
%Fig. 2), onde os usuários estão conectados a quatro repositórios por quatro caminhos não disjuntos. Usuários no nó de acesso exclusivo 

\vspace{0.8cm}
\begin{figure*}[htpb]
	\centering
	\includegraphics[width=0.75\textwidth]{img/exp-multi-lvl}
% 	\vspace{-1cm}
	\caption{Cenários de distribuição de streaming de vídeo adaptável para nós multiníveis.}
	\label{fig:scenario-arch}
\end{figure*}
%\vspace{0.8cm}

As Figuras 4(a) e 5(a) mostram a taxa de bits media com 20 e 40 usuários, respectivamente. No cenário com 20 usuários~(Figura 4(a)) as três heurísticas ABR apresentam um impacto semelhante. Enquanto que, intuitivamente, o provisionamento do video DASH entre os níveis 1 e 2 apresentam uma diferença estritamente superior, em contrapartida, o nível 3 apresenta uma disparidade significativa em relação aos outros níveis. Para o cenário com 40 usuários a taxa de bits médio entre níveis apresentam uma diferenças semelhantes em ordem de grandeza. 

As Figuras 4(b) e 5(b) mostram o tempo de interrupção total com 20 e 40 usuários, respectivamente. Ao contrario da taxa de bit media, aqui nós podemos observar uma diferença em relação aos mecanismos ABR no cenário com 20 usuários~(Figura 4(a)), principalmente, quando o video é provisionado por um servidor no nível com mais saltos~(nível 1). Este comportamento é suavizado no cenário com 40 usuários, onde existe uma pequena diferença ente os mecanismos de decisão no nível 2.

É importante notar que apesar da taxa de bits media entre os mecanismos ABR serem semelhantes, as interrupções que eles podem gerar são significativamente diferentes, o que impacta diretamente no QoE dos usuários. Outro ponto a ser observado é a taxa de bits media entre diferentes níveis. No cenário com 20 usuários o nível 3 conseguiu atingir o que necessário para obter a maior representação de taxa bits, desta forma, operadores de telefonia podem buscar a implantação de caches durante horários de pico para prover a melhor experiencia ao usuário.

%entre os 
% diferença de npirveis há o mesmo pode ser observado no segundo cenario co 40 usuários. Uma diferença interessante 
%
%
%Avaliamos o impacto que uma arquitetura multinível de streaming de vídeo adaptativo em uma topologia com tres níveis com um único caminho, apresentada na Fig. 5. Os links possuem comprimento de possui um taxa de dados de 10 MBps e cada AP .


\vspace{0.8cm}
\begin{figure}[htb]
  \centering
    \subfloat[Taxa de bits medio.]
    {\includegraphics[width=.45\textwidth]{graphs/boxplot-avgBt-10u}
    \label{fig:lte-handover}}
  \hfil \hspace{1cm}
    \subfloat[Tempo total de interrupções.]
    {\includegraphics[width=.45\textwidth]{graphs/boxplot-avgStalls-10u}
    \label{fig:dmm-proposal}}
  \caption{Resultados da taxa de bits media e tempo total de interrupções para uma rede com 20 usuários, 10 em cada AP.}
  \label{fig:dmm}
\end{figure}

\begin{figure}[htb]
  \centering
    \subfloat[Taxa de bits medio.]
    {\includegraphics[width=.45\textwidth]{graphs/boxplot-avgBt-20u}
    \label{fig:lte-handover}}
  \hfil \hspace{1cm}
    \subfloat[Tempo total de interrupções.]
    {\includegraphics[width=.45\textwidth]{graphs/boxplot-avgStalls-20u}
    \label{fig:dmm-proposal}}
  \caption{Resultados da taxa de bits media e tempo total de interrupções para uma rede com 40 usuários, 20 usuários em cada AP.}
  \label{fig:dmm}
\end{figure}
\clearpage
\section{Research topics to be investigated}
\label{ch:proposal}

This section presents some main ideas for the continuity of this doctoral
research project. \autoref{sec:topics} brings the research topics to be
addressed while \autoref{sec:timetable} shows the work plan and timetable.

%=============================================================================%
\subsection{Research topics}
\label{sec:topics}


Devido ao compartilhamento de ambientes de rede, a natureza do \textit{best-effort} da infraestrutura da Internet, e o comportamento egoísta totalmente isolado dos jogadores da HAS. é difícil para streams de vídeo satisfazer os três objetivos descritos abaixo. Desde de que existam multiplos clientes concorrentemente competindo entre eles um limitado comprimento de onda, as soluções existentes para clientes dirigido a HAS não trabalham vem e sofrem muitos problemas com perda de pacotes, flutação de comprimento de onda, oscilações na qualidado e troca de taxas de bits~(Instabiliddade do vídeo mostrada ba Figura~\TODO{Falta gerar gráfico}), QoE indesejavel e compartilhamento de recursos da rede, recursos da rede inutilizados ou sobrescritos do qual 'will adversely affect the viewer's QoE'. Alem disso, estes problemas tem sido confirmados em experimentos recentas. 

The research topics to be addressed are divided into three topics, which are
detailed in Sections~\ref{subsec:distributed} through \ref{subsec:handover}.
Each topic encompasses the motivation, goals, and objectives.

%-----------------------------------------------------------------------------%
\subsubsection{Scalable controller architecture}
\label{subsec:}

Este projeto tem como objetivo projetar uma entrega de vídeo baseada em DASH confiável e de alta qualidade para ser usada em ambientes de cidades inteligentes~\cite{gamaUCC2019, KreuzbergerWorkshop2016}. O esquema proposto aproveitará várias tecnologias relacionadas à rede, como Cloud, Fog e Edge Computing, além de posicionamento e encadeamento inteligente de serviços.% A Figura~\ref{fig:scenario-arch} descreve, no lado esquerdo, uma arquitetura de rede de várias camadas, composta por um conjunto heterogêneo de dispositivos e aplicativos que utilizam recursos de computação distribuídos por meio de uma tecnologia de comunicação de acesso múltiplo, como 5G e WiFi. Este projeto propõe estender o streaming de vídeo DASH para suportar conectividade simultânea de caminhos múltiplos~\cite{poliakovPHD2018, Velasquez2018}.

%O lado direito da Figura~\ref{fig:scenario-arch} descreve parte dos parâmetros que devem ser avaliados para definir quais serviços de vídeo precisam ser implantados, juntamente com a camada mais adequada para implantar cada um deles. Observe que os parâmetros na camada mais inferior para feedback diferem dos de outras camadas. Inicialmente, os parâmetros avaliados considerados incluem o perfil do usuário, a carga da célula local, a qualidade do link, a complexidade do movimento dos vídeos e também detalhes inteligentes da cidade, como a localização e a rota rastreada no caso de usuários com mobilidade. Alguns dos nós podem ser estacionários, mas outros podem variar de padrões de baixa a alta mobilidade, que podem ser levados em consideração para melhorar a qualidade da entrega de vídeo.

\vspace{0.8cm}
\begin{figure*}[htpb]
	\centering
	\includegraphics[width=0.7\textwidth]{img/fig-intro.png}
% 	\vspace{-1cm}
	\caption{DASH-based Adaptive Multimedia Delivery System for Cloud/Fog Nodes.}
	\label{fig:scenario-arch}
\end{figure*}

Dada a arquitetura do modelo de serviço e ambiente de nuvem/borda de várias camadas mencionada, este trabalho tem como objetivo abordar algumas das seguintes questões de pesquisa:~\textit{i)} Como determinar as melhores camadas para alocação de serviços de vídeo?~\textit{ii)} Como os pedaços de vídeo não solicitados devem ser distribuídos na hierarquia de borda/nuvem, considerando as informações de localização do usuário e estimativas sobre a localização futura do usuário em tempo real?~\textit{iii)} Como facilitar o streaming de vídeo através de várias fontes simultaneamente?~\textit{iv)} Como os algoritmos de adaptação à taxa de bits podem ser afetados pelo tamanho do pedaço de vídeo em uma arquitetura de várias camadas?

\subsection{Multimedia Delivery System Schemes}
% Selection Algorithms

\subsection{Game Theory Based Adaptative Bit Rate Scheme}

% This project aims to design a reliable and high-quality DASH-based video delivery to be used in Smart City Environments~\cite{gamaUCC2019, KreuzbergerWorkshop2016}. The proposed scheme will take advantage of several network-related technologies such as Cloud, Fog, and Edge Computing, as well as intelligent service placement and chaining. Figure~\ref{fig:scenario-arch} depicts, on its left-hand side, a multi-tier network architecture, which is composed of a heterogeneous set of devices and applications using distributed computing resources through a multi-access communication technology, such as 5G and WiFi. This project proposes to extend DASH video streaming to support simultaneous multipath connectivity~\cite{poliakovPHD2018, Velasquez2018}.

% The right-hand side of Figure~\ref{fig:scenario-arch} describes part of the parameters that should be assessed to define which video services are needed to be deployed along with the most suitable tier to deploy each of them. Note that the parameters in the bottommost tier for feedback differ from those of other tiers. Initially, the assessed parameters considered include the user's profile, the load of the local cell, the link quality, the motion complexity of the videos, and also smart city details such as the location and the traced route in case of users with mobility. Some of the nodes can be stationary, but others can range from low to high mobility patterns, which can be taken into account to improve quality of video delivery.

% \vspace{0.8cm}
% \begin{figure*}[htpb]
% 	\centering
% 	\includegraphics[width=1.0\textwidth]{images/scenario_incomplete}
% % 	\vspace{-1cm}
% 	\caption{DASH-based Adaptive Multimedia Delivery System in an Smart City Environment.}
% 	\label{fig:scenario-arch}
% \end{figure*}

% Given the aforementioned multi-tier edge/cloud environment and service model architecture, this work aims to tackle some of the following research questions:~\textit{i)} How to determine the best tiers for video services placement?~\textit{ii)} How unsolicited video chunks should be distributed in the edge/cloud hierarchy considering user location information and estimates on future user location in real time? ~\textit{iii)} How to facilitate video streaming through multiple sources concurrently?~\textit{iv)} How bitrate adaptation algorithms can be impacted by video chunk size in a multi-tiered architecture?


% Neste trabalho, queremos construir um sistema CDN na fog com o uso de cache e sobreposição de rede para para streaming de video ao vivo, e otimizar sua topologia de forma adaptativa para minimizar a latência de reprodução média e melhorar a entrega do fluxo de forma oportuna. A latência de reprodução é a diferença entre o tempo de reprodução (ponto de reprodução) na origem de mídia e em um nó.

% \vspace{1cm}
% \noindent
% \textbf{Modelagem de propriedades de balanceamento de carga e desempenho de cache em sistemas de Névoa-Nuvem multicamadas.}

%-----------------------------------------------------------------------------%
\subsection{Medidas objetivas}


%-----------------------------------------------------------------------------%
\subsubsection{Scalable controller architecture}
\label{subsec:distributed}

\emph{The goal of this research topic is} to improve the current controller
model toward a scalable architecture. \autoref{fig:distributed-controller}
shows a possible distributed \ac{EPC} controller architecture, where the
\acp{eNB} are enhanced with a local controller and switch that are used for
traffic classification purposes at the network edge. A centralized topology
controller can independently handle topology-related questions (like traffic
routing) in the backhaul network. Finally, a specialized \ac{QoS} controller
would be in charge of admission control and \ac{QoS} management, communicating
with edge and topology controller. Theses controllers can be assisted by
applications, like network mapper and routing and a more sophisticated
admission control software. 
\emph{The objectives for this research topic are:}
\begin{itemize}
  \item Determine the controller functionalities that can be performed
  independently of each other, to effectively proposed a distributed controller
  architecture;

  \item Model the inter-controller communication for the distributed controller
  architecture, reducing the amount of information exchanged between
  controllers;

  \item Evaluate possible standardized east/westbound interfaces for controller
  interaction;

  \item Implement and evaluate the proposed architecture in the \ac{ns-3}
  simulator, performing scalability tests with large numbers of \acp{eNB} and
  \acp{UE} in the network.
\end{itemize}

%-----------------------------------------------------------------------------%
\subsubsection{Traffic offloading in \acsp{HetNet}}
\label{subsec:heterogeneous}

As the wireless link efficiency is approaching its fundamental limits, further
improvements in cellular system spectral efficiency are only possible by
increasing the node deployment density. As observed by \citet{Damnjanovic2011},
challenges associated with the deployment of traditional macro base stations
can be overcome by the utilization of base stations with lower transmit power.
A network that consists of a mix of macro cells and low-power nodes, where some
may be configured with restricted access and some may lack wired backhaul, is
referred to as a \acf{HetNet}. \autoref{fig:hetnet} exemplifies a \ac{HetNet}
with a macro cell, some metro and picocells used for relay, some network
operator deployed low-costs indoor and outdoor small cells, and user deployed
very low costs for indoor environment.

\begin{figure}[htb]
  \centering
  \includegraphics[scale=0.55]{hetnet}
  \caption{A \acs{HetNet} example scenario~\cite{Qualcomm2014}.}
  \label{fig:hetnet}
\end{figure}

The future 5G architecture must provide a communication environment able to
overcome the infrastructure shortcomings of current networks. Networks will
become much denser with many more cells with decreasing size as well as direct
device-to-device communication. Small cells improve capacity and cellular
coverage with lower cost compared to macrocells, and they are expected to carry
the majority of traffic. \citet{Pierucci2015} says that while bringing the base
station closer to the user, it is possible to promote lower power use and more
energy efficient communications. In the opinion of \citet{Einsiedler2015}, the
vision is to provide functional convergence of network control to handle 5G,
4G, older access technologies, and Wi-Fi; enabling a flexible and efficient
support and deliver all types of applications.

As claimed by \citet{Tomici2015}, network operators have shown more interest in
deploying integrated small cell and Wi-Fi accessing networks to accommodate the
increasing demand for bandwidth caused by widespread wireless data usage.
However, the 3GPP interworking architecture forces the inter-system handover to
happen solely at the \ac{P-GW}, which may cause unnecessary burdens on the
mobile core network, especially with the expected large number of small cell
and Wi-Fi accessing network deployment.

Within an \ac{SDN} architecture, the user data plane can be distributed to
allow local offloading of user data traffic, regulated by the (partially)
logically centralized network controller. An example of traffic offloading
solution is the one proposed by \citet{Ghazisaeedi2013}, where a switching
system is used to redirect \ac{HTTP} traffic from \ac{S-GW} to the Internet.
This technique can decrease the traffic load over the mobile core network. In
\citet{Tomici2015}, the authors suggest three new \ac{LTE}-\ac{WLAN}
integration architectures for inter-system handover, which are based on having
a local integration point. For each of these integrated architectures, the
authors propose a network-initiated inter-system handover mechanism between the
\ac{eNB} and \ac{WLAN} \ac{AP}. The handover mechanisms are efficient due to
limited interaction with the \ac{P-GW} in the core network.

\emph{The goal of this research topic is} to design an algorithm responsible
for offloading \ac{LTE} traffic over available Wi-Fi networks. Assuming that
many Wi-Fi hotspots are deployed by the mobile operators in public urban areas,
the algorithm focus is to trigger offloading decisions based on the current
traffic load in the backhaul and core networks.
\emph{The objectives for this research topic are:}
\begin{itemize}
  \item Compare current approaches for traffic offloading in mobile networks,
  identifying the benefits and drawbacks of each solution;

  \item Propose an improved offloading decision solution for \ac{EPC}, using
  Wi-Fi as enabler technology;

  \item Implement and evaluate the proposed solutions in the \ac{ns-3}
  simulator, performing tests on scenarios with different access networks.
\end{itemize}

%-----------------------------------------------------------------------------%
\subsubsection{Distributed mobility management}
\label{subsec:handover}

Mobility management refers to a set of mechanisms to keep ongoing sessions
continuity while a mobile user changes its mobility anchor point in the
network. In the \ac{LTE} architecture, mobility management solutions rely on
centralized mobility anchor entities (\ac{S-GW} and \ac{P-GW}), which are in
charge of both mobility-related control plane and user data forwarding.
According to \citet{Valtulina2014}, this centralized approach makes mobility
management prone to several performance limitations such as suboptimal
routing, low scalability, potential single point of failure and the lack of
granularity for the mobility management service.

Nowadays, \acf{DMM} is considered as a promising solution to solve these above
challenges. Several proposals tried to redesign mobile network architecture by
leveraging \ac{SDN} and OpenFlow with the support of \ac{DMM}.
\citet{Karimzadeh2014} discuss the \ac{DMM} in \ac{LTE} systems composed of
virtual gateways running on the cloud. \citet{Kuklinski2014b} discuss the
evolution of mobility management mechanisms in mobile networks and how \ac{SDN}
can be applied to efficiently handle mobility in the context of future 5G
networks. In the CROWD architecture~\cite{Ahmad2013a}, a \ac{DMM} gateway
replaces current anchor points, and mobility management functions are executed
by control applications. \citet{Gurusanthosh2013} propose the \acf{SDMA} in
\ac{LTE} backhaul access networks, including detailed handover procedure with
dynamic anchor element choice, which is based on \ac{UE} location and path
changes during handover. \citet{Mahmoodi2014} introduce a distributed \ac{SDN}
control plane with mobility management as an example where the controller
becomes responsible for handover procedures, reducing the power consumption of
the \ac{UE} and the signaling message overhead between entities at the backhaul
side.

Besides, mobility management may no longer be exclusively triggered by radio
quality, but also by network management decisions~\cite{Rost2014}. Several
redesigned handover procedures are proposed by \citet{Zhang2015b} for 5G ultra
dense networks. They take advantage of control and user-plane separation,
taking into consideration of mobility, stability, energy efficiency and
realizability.

Particularly, handover performance can severely impact the \ac{QoS} in
\acp{HetNet}. The increasing number of small nodes bring cells closer and there
is no more obvious boundary between them, enhancing the capacity per area by
the space division multiplexing. \citet{Sun2015} propose intelligent schemes
to handle the dynamic environment of \acp{HetNet}. The authors identify some
key problems in 5G \acp{HetNet} as traffic control, load balancing, density
prediction, and resource allocation. Then, they develop a number of smart
schemes to overcome these challenges.

\emph{The goal of this research topic is} to distribute the mobility management
computation intensive task among the \ac{MME} and a number of controllers nodes
in the network, avoiding a centralized operation. The handover processing can
also be assisted by a sophisticated software application. With this approach,
it is possible to reduce signaling overhead over backhaul and core network and
speed up handover procedures. The development of such distributed solution will
be guided by group handover in heterogeneous environments, when a group of
users may change the access networks simultaneously or within short
time~\cite{Chowdhury2012}.

Another opportunity is to explore \ac{SDN} flexibility and move the anchor
point to the OpenFlow backhaul network, considering that when a \ac{UE} is
handed over to a new access point, in most cases a large part of the path that
the traffic takes in the backhaul network would be the same before and after
the handover. \autoref{fig:lte-handover} shows how \ac{GTP} tunnels are handled
during a normal handover procedure in \ac{LTE} networks, and
\autoref{fig:dmm-proposal} exemplifies how the proposed distributed mobility
management solution can handle the \ac{GTP} tunnels, following some ideas from
the \ac{SDMA}. Nonetheless, while \ac{SDMA} demands changes in the standard
procedures with no backward compatibility, the proposed solution must be
designed to offer interoperability with existing networks.

\begin{figure}[htb]
  \centering
    \subfloat[Centralized anchor.]
    {\includegraphics[width=.25\textwidth]{lte-handover}
    \label{fig:lte-handover}}
  \hfil \hspace{1cm}
    \subfloat[Distributed anchor.]
    {\includegraphics[width=.25\textwidth]{dmm-proposal}
    \label{fig:dmm-proposal}}
  \caption{\acs{SDN}-enabled \acs{LTE} network topology.}
  \label{fig:dmm}
\end{figure}

\emph{The objectives for this research topic are:}
\begin{itemize}
  \item Detailed study of \ac{LTE} handover procedures to propose a suitable
  distribution among \ac{MME} and available OpenFlow controllers, considering
  the interoperability with \ac{3GPP} standards;

  \item Examine available solutions for load balancing in mobile networks,
  which can be used to trigger handover procedures in the distributed
  architecture;

  \item Analyze existing solutions for vertical handover in \acp{HetNet},
  taking care of handovers between overlapping macro and small cells;

  \item Assess approaches for rerouting tunnels in OpenFlow backhaul network,
  looking for an optimal anchor element;

  \item Implement and evaluate the proposed \ac{DMM} in the \ac{ns-3}
  simulator, performing handover tests with groups of \acp{UE} between
  \acp{eNB}.
\end{itemize}


%=============================================================================%
\subsection{Work plan and timetable}
\label{sec:timetable}

\autoref{tab:timetable} presents the timetable for this doctoral research
project. The activities referenced in the timetable are listed below and
comprises both the developed work introduced in \autoref{ch:developed} and the
topics to be investigated from \autoref{sec:topics}. The activities that are
already developed are identified by the symbol~\,\m. Meanwhile, the symbol
\,\x\, identifies the expected time for carrying out the planned activities.

\begin{enumerate}
	\itemsep0pt
  \item Detailed study on \ac{SDN} and \ac{LTE} integration;
  \item Evaluation of available software tools for performance analysis;
  \item Development of the OpenFlow 1.3 module for \ac{ns-3};
  \item Proposal of the \ac{SDN}-enabled \ac{LTE} network;
  \item OpenFlow \ac{EPC} controller for traffic routing and bearer admission
        control;
  \item \ac{LTE} \ac{QoS} realization with OpenFlow protocol;
  \item Doctoral qualifying exam writing and defense;
  \item Scalable controller architecture proposal;
  \item Traffic offloading in heterogeneous networks;
  \item Distributed mobility management solutions;
  \item Thesis writing and defense.
\end{enumerate}

\begin{table}[htb]
  \renewcommand{\arraystretch}{1.4}
  \caption{The timetable for this doctoral research project.}
  \label{tab:timetable}
  \tiny
  \centering
  \begin{tabular}{c|c|cccc|cccc|cccccc|c}
    \toprule
    & {\bf 2013}
    & \multicolumn{4}{c|}{{\bf 2014}}
    & \multicolumn{4}{c|}{{\bf 2015}}
    & \multicolumn{6}{c|}{{\bf 2016}}
    & {\bf 2017} \\

    & {\it Oct} & {\it Jan} & {\it Apr} & {\it Jul} & {\it Oct} & {\it Jan} &
    {\it Apr} & {\it Jul} & {\it Oct} & {\it Jan} & {\it Mar} & {\it May} &
    {\it Jul} & {\it Sep} & {\it Nov} & {\it Jan} \\

    & {\it Dec} & {\it Mar} & {\it Jun} & {\it Sep} & {\it Dec} & {\it Mar} &
    {\it Jun} & {\it Sep} & {\it Dec} & {\it Feb} & {\it Apr} & {\it Jun} &
    {\it Aug} & {\it Oct} & {\it Dec} & {\it Feb} \\
    \hline % \midrule
    \arrayrulecolor{lightgray}

    %        |2013|       2014        |       2015        |            2016             |2017
    %        | 08 | 01   04   07   10 | 01   04   07   10 | 01   03   05   07   09   11 | 01
    %        | 12 | 03   06   09   12 | 03   06   09   12 | 02   04   06   08   10   12 | 02
    {\bf 01} & \m & \m &    &    &    &    &    &    &    &    &    &    &    &    &    &    \\ \hline
    {\bf 02} &    & \m &    &    &    &    &    &    &    &    &    &    &    &    &    &    \\ \hline
    {\bf 03} &    & \m & \m & \m & \m &    &    &    &    &    &    &    &    &    &    &    \\ \hline
    {\bf 04} &    &    &    & \m & \m & \m &    &    &    &    &    &    &    &    &    &    \\ \hline
    {\bf 05} &    &    &    &    &    & \m & \m & \m &    &    &    &    &    &    &    &    \\ \hline
    {\bf 06} &    &    &    &    &    &    &    & \m & \m &    &    &    &    &    &    &    \\ \hline
    {\bf 07} &    &    &    &    &    &    &    &    & \x & \x &    &    &    &    &    &    \\ \hline
    {\bf 08} &    &    &    &    &    &    &    &    & \x & \x & \x &    &    &    &    &    \\ \hline
    {\bf 09} &    &    &    &    &    &    &    &    &    & \x & \x & \x & \x & \x &    &    \\ \hline
    {\bf 10} &    &    &    &    &    &    &    &    &    &    &    & \x & \x & \x & \x &    \\ \hline
    {\bf 11} &    &    &    &    &    &    &    &    &    &    &    &    &    & \x & \x & \x \\

    \arrayrulecolor{black}
    \bottomrule
  \end{tabular}
\end{table}



%=============================================================================%
\section{Motivação}

%Este projeto propõe o uso da hierarquia de borda/nuvem para projetar um streaming de vídeo DASH cooperativo nas Smart Cities, implantando o serviço de cache para oferecer melhor qualidade de experiência~(QoE) para os usuários finais.
%At the same time, video streaming services represent the majority of the internet traffic, and according to Cisco forecasts\footnote{Cisco Visual Networking Index: Global Mobile Data Traffic Forecast Update. Link:~\url{http://shorturl.at/hjAZ1}. Accessed: July 29, 2019.}, in 2021 70\% of all internet traffic will be dominated by video streaming. This includes current video services as well as innovative services such as cloud gaming and future consoles (e.g. Google Stadia), whereas for mobile devices this estimate represents 78\% of all mobile data traffic. To accommodate video traffic, a good cloud-level architecture partially solves some issues related to the live stream and Video on Demand~(VoD) services. However, a centralized cloud service introduces some issues such as higher latency and core network congestion. Therefore, to improve video services, it is of paramount importance to properly distribute video streams according to their requirements: a cloud gaming infrastructure is an interactive service that needs reduced delays (a few milliseconds), while a non-interactive VoD delivery can tolerate higher delays without impairing quality of experience. A proper management and orchestration of video delivery over the Internet is core to the smooth co-existence of heterogeneous video services. This project proposes the use of edge/cloud hierarchy to design a cooperative DASH video streaming in Smart Cities, deploying cache service to offer improved Quality of Experience~(QoE) for end-users.

%*********** about IOT ******************


%***********Objectives of the proposal***********
1. How to deliver high quality streaming using in-Network coding and caching?
2. How to enable the multipath and multicast capabilities for HAS in future networks like ICN?
3. What is the impact of these networks on HAS decisions?
4. What is the benefits that edge computing can add in HAS over the future networks?
5. What is the deployment cost?


\section{Problem Statement}

The most of the exiting HAS delivery solutions, and their ABR schemes have four major shortcomings which are summarized as follows:

1. Multi-player problem.

2. Bandwidth fluctuation problem.

3. Quality fairness and heterogeneous system problem.

4. Trade-off between QoE metrics and ABR objectives problem.

\clearpage
\section{Final remarks}
\label{ch:remarks}

This research proposal is intended to explore the potentials of \acl{SDMN}. 
% ----------------------------------------------------------------------------
% How the SDN paradigm can be actually used to expand existing networks, moving
% away the vendor dependence but sustaining the performance achieved by
% dedicated hardware? 
% ----------------------------------------------------------------------------
Through a comprehensive literature review, it is possible to observe how the
existing solutions can improve mobile networks toward \ac{SDMN}. Different
approaches are used to simplify the equipment and increase flexibility due to
\ac{SDN} centralized control.
% ----------------------------------------------------------------------------
% How the proposed solutions are evaluated and what is the effective adoption
% of these solutions? 
% ----------------------------------------------------------------------------
There are many proposals in the area, but the performance evaluation processes
for new solutions is not uniform. Some solutions have no performance validation
while other works evaluate their proposals using small software test bed. To
overcome this need, a new OpenFlow module was developed, allowing \ac{ns-3}
simulations in this area. In addition, a new simulation scenario has been
proposed, and the current centralized controller architecture is to be
distributed among local agents in the direction of a scalable solution.
% ----------------------------------------------------------------------------
% How to manage upcoming heterogeneous 5G networks, considering different radio
% technologies and the explosion of connections? 
% How to effectively simplify mobility management in current architectures,
% exploiting the SDN centralized network view?
% ----------------------------------------------------------------------------
Taking into account that future 5G networks will become much denser with many
more cells, another item of interest in this research is how to perform user
handover and traffic offloading between different cells, and even between
different technologies. To this end, it is proposed a distributed mechanism for
dealing with mobility management decisions.



\setlength{\bibsep}{7.5pt}
\singlespacing
\footnotesize{
  \bibliographystyle{unsrt}
  \bibliography{references}
}

\end{document}

\grid
\grid
\grid
\grid
