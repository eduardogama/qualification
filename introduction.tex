\section{Introduction}
\label{ch:introduction}

%******* Introduction of the Dash technology and Cloud/Fog networks****


% ----------------------------------------------------------------------------
% What is the current context in mobile network in terms of traffic and users? 
% ----------------------------------------------------------------------------
Os serviços de streaming de vídeo representam a maior parte do tráfego da Internet. De acordo com as previsões da Cisco~\footnote{Visual Networking Index: atualização global de previsão de tráfego de dados móveis. Link:~\url{http://shorturl.at/hjAZ1}. Acesso em: 29 de julho de 2019.}, em 2021, 70\% de todo o tráfego da Internet será dominado por streaming de vídeo. Isso inclui os serviços de vídeo atuais, bem como serviços inovadores de jogos na nuvem e futuros consoles~(por exemplo, Google Stadia), enquanto para dispositivos móveis essa estimativa representa 78\% de todo o tráfego de dados móveis. Essa tendência impõe novos desafios no fornecimento de vídeos com a melhor \acl{QoE}~(QoE) na Internet atual, originalmente projetada levando em consideração o modelo de \textit{best-effort} para a transmissão de dados.

% A Fig. 1 Mostra ao longo de cada ano o crescimento no consumo de banda, estes tipos de cenários impõem significantes desafios para a distribuição de vídeo sobre as futuras gerações da Internet.



Atualmente, esses conteudos multimidia utilizam serviços de Streaming Adaptativo sobre HTTP~(HAS), este paradigma trata o conteúdo da mídia como qualquer outro conteúdo comum da Web e o entregou em pequenos pedaços pelo protocolo HTTP.
Os serviços de transmissão de vídeo têem seu bitrate adaptados dinâmicamente.% As soluções HAS utilizam os protocolos HTTP na aplicação e TCP na camada de transporte como ilustrados na figura 1b. 
em Streaming Adaptativo Dinâmico sobre HTTP~(DASH) são amplamente adotado por provedores de vídeo como Google, Netflix, Akamai HD e outros, nos quais o player de vídeo do cliente pode escolher dinamicamente o nível de taxa de bits de acordo com a largura de banda disponível percebida.
As soluções HAS empregam adaptação dinâmica em relação às condições de rede variáveis para fornecer uma experiência de streaming contínua~(ou pelo menos mais suave). Depois que um arquivo de mídia~(ou fluxo) estiver pronto a partir de uma fonte, ele será preparado para transmissão em fluxo antes de ser publicado em um servidor HTTP padrão e pronto para uso.

%HAS uses HTTP as the application and TCP as the transport-layer protocol as illustrated in Figures 1.2b and 1.3, and clients pull the data from an HTTP server. HAS solutions employ dynamic adaptation with respect to varying network conditions to provide a seamless (or at least smoother) streaming experience. Once a media file (or stream)is ready from a source, it is prepared for streaming before it is published to a standard, off-the-shelf HTTP server. The original file/stream is partitioned into segments (also calledchunks) of equi-length playback time, and multiple versions (also called representations)of each segment are generated that vary in bitrate/resolution/quality using an encoder ora transcoder (i.e., H.264, H.265, etc.). Moreover, the server generates an index file, whichis a manifest that lists the available representations including HTTP uniform resourcelocators (URLs) to identify the segments along with their availability times. During atypical HAS session, the client first receives the manifest that contains the metadata forthe video, audio, subtitles, etc., and then constantly measures certain parameters such as the available network bandwidth, buffer status, and battery and CPU levels. According tothese parameters, the HAS client repeatedly fetches the most suitable next segment amongthe available representations from the server. Table 1.1 compares the main characteristicsof the traditional streaming and HAS systems.

% ----------------------------------------------------------------------------
% What carries have been doing to address the increasing traffic?
% ----------------------------------------------------------------------------
Em cenários para as futuras gerações da Internet, tecnologias de redes sem fio, como, redes 5G, mobile-edge computing~(MEC), redes definida por software~(SDN).
%Esses fatores demonstram como o futuro da demanda da Internet pode se tornar muito grande para os ISPs e os avanços tecnológicos nas telecomunicações nas quais eles dependem. 
Atualmente, tradicionais serviços de Streaming de vídeo sob Demanda~(VoD) são projetados para distribuir conteúdo multimídia por grandes data centers na nuvem. Esses sistemas geralmente usam um conjunto de servidores em que o tráfego passa pel núcleo da rede; além disso, a conexão dos dispositivos é feita por usuários estáticos e por links da Internet estáveis [?]. Dessa forma, a criação desses sistemas no nível da nuvem resolve parcialmente os problemas de escalabilidade, disponibilidade e interoperabilidade, mas ao mesmo tempo, apresenta novos desafios (por exemplo, maior latência e congestionamento da rede principal) [?]. Vários trabalhos na literatura destacam a computação de nevoa/borda para lidar com as novas demandas de tráfego que estão surgindo. Onde datacenters com menos capacidade de processamento e armazenamento conseguem prover serviços/virtualização mais próximos ao usuário final. Assim, a borda da rede pode fornecer taxas de latência que a nuvem não conseguem alcançar de outra forma [?], [?].

%These two factors demonstrate how the future of Internet demand may become toooverwhelming for both ISPs and technological advancements in telecoms they rely on. Currently, the traditional VoD services are designed to distribute multimedia content by large data centers at the cloud. These systems usually use a set of servers where the traffic passes through the core network, beyond that, the devices connection are made by static end-users and Internet links stable [?]. This way, the inception of these systems at cloud-level partially solves the scalability, availability, and interoperability issues, but at the same time, introduces new ones (e.g., higher latency and core network congestion) [?]. Several works in the literature have highlighted fog/edge computing to deal with new traffic demands that are emerging. Where datacenters with less processing and storage capacity can provision virtualization/services closer to the end-user. Thus, the network edge can provide latency rates that the cloud is not able to reach otherwise [?], [?].


% ----------------------------------------------------------------------------
% What are the problems with existing networks?
% ----------------------------------------------------------------------------

%Para melhorar os serviços de vídeo, é de suma importância distribuir adequadamente os fluxos de vídeo de acordo com seus requisitos: uma infraestrutura de jogos em nuvem é um serviço interativo que precisa de atrasos reduzidos~(alguns milissegundos), enquanto uma entrega de VoD não interativa pode tolerar maior atrasos sem prejudicar a qualidade da experiência. Um gerenciamento e orquestração adequados da entrega de vídeo pela Internet é essencial para a coexistência suave de serviços de vídeo heterogêneos.

%Currently, the traditional VoD services are designed to distribute multimedia content by large data centers at the cloud. These systems usually use a set of servers where the traffic passes through the core network, beyond that, the devices connection are made by static end-users and Internet links stable [?]. This way, the inception of these systems at cloud-level partially solves the scalability, availability, and interoperability issues, but at the same time, introduces new ones (e.g., higher latency and core network congestion) [?].

%Esse novo paradigma, ao qual nos referimos como HTTP Adaptive Streaming (HAS), tratava o conteúdo da mídia como qualquer outro conteúdo comum da Web e o entregava em pequenos pedaços pelo protocolo HTTP. O HAS se tornou rapidamente a abordagem dominante para o streaming de vídeo devido à sua adoção pelos principais provedores de serviços e conteúdo. A entrega de vídeo pela Internet pública também é conhecida como streaming de vídeo over-the-top (OTT), pois o conteúdo ou o provedor de serviços de streaming geralmente é diferente do provedor de rede. O surgimento do HAS e de novos dispositivos móveis principalmente para usuários finais, com altos recursos de processamento e renderização, teve um papel fundamental no crescimento do tráfego de streaming de vídeo


% ----------------------------------------------------------------------------
% Which technologies can be used to improve future networks?
% ----------------------------------------------------------------------------

% ----------------------------------------------------------------------------
% How the SDN parading is contributing to current networks?
% ----------------------------------------------------------------------------

% ----------------------------------------------------------------------------
% Why and how are we contributing in this topic?
% ----------------------------------------------------------------------------

%In an attempt to answer above questions, this ongoing doctoral research project discusses how the \ac{SDN} paradigm and the OpenFlow protocol can be integrated to the existing 4G \ac{LTE} networks to provide new solutions for some of the aforementioned problems. As contributions already developed, this project first ...
 
Este projeto tem como objetivo projetar uma entrega de vídeo baseada em DASH confiável e de alta qualidade para ser usada em ambientes de cidades inteligentes [1, 2]. O esquema proposto aproveitará várias tecnologias relacionadas à rede, como Cloud, Fog e Edge Computing, além de posicionamento e encadeamento inteligente de serviços. A Figura 1 mostra, no lado esquerdo, uma arquitetura de rede de várias camadas, composta por um conjunto heterogêneo de dispositivos e aplicativos usando recursos de computação distribuídos por meio de uma tecnologia de comunicação de acesso múltiplo, como 5G e WiFi. Este projeto propõe estender o streaming de vídeo DASH para oferecer suporte à conectividade multipath simultânea [3, 4].

% ----------------------------------------------------------------------------
% How is this document organized?
% ----------------------------------------------------------------------------
This document is organized as follows: \autoref{ch:background} presents some
background concepts on \ac{SDN} and \ac{LTE} architectures. Then,
\autoref{ch:integration} gives an overview of several proposals for \ac{SDMN}.
It focuses on the use of \ac{SDN} solutions in the backhaul and core network,
comparing some related works and highlighting major open challenges in the
topic. \autoref{ch:developed} brings the work carried out so far, which will
assist the development of the proposed contributions that are detailed in
\autoref{ch:proposal}. Finally, \autoref{ch:remarks} closes the document with
some final remarks.

