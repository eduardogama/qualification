\clearpage
\section{Final remarks}
\label{ch:remarks}

This research proposal is intended to explore the potentials of \acl{SDMN}. 
% ----------------------------------------------------------------------------
% How the SDN paradigm can be actually used to expand existing networks, moving
% away the vendor dependence but sustaining the performance achieved by
% dedicated hardware? 
% ----------------------------------------------------------------------------
Through a comprehensive literature review, it is possible to observe how the
existing solutions can improve mobile networks toward \ac{SDMN}. Different
approaches are used to simplify the equipment and increase flexibility due to
\ac{SDN} centralized control.
% ----------------------------------------------------------------------------
% How the proposed solutions are evaluated and what is the effective adoption
% of these solutions? 
% ----------------------------------------------------------------------------
There are many proposals in the area, but the performance evaluation processes
for new solutions is not uniform. Some solutions have no performance validation
while other works evaluate their proposals using small software test bed. To
overcome this need, a new OpenFlow module was developed, allowing \ac{ns-3}
simulations in this area. In addition, a new simulation scenario has been
proposed, and the current centralized controller architecture is to be
distributed among local agents in the direction of a scalable solution.
% ----------------------------------------------------------------------------
% How to manage upcoming heterogeneous 5G networks, considering different radio
% technologies and the explosion of connections? 
% How to effectively simplify mobility management in current architectures,
% exploiting the SDN centralized network view?
% ----------------------------------------------------------------------------
Taking into account that future 5G networks will become much denser with many
more cells, another item of interest in this research is how to perform user
handover and traffic offloading between different cells, and even between
different technologies. To this end, it is proposed a distributed mechanism for
dealing with mobility management decisions.

