\clearpage
\section{Considerações finais}
\label{ch:remarks}

Através de uma revisão da literatura, pode-se observar que existem muitas propostas na área, no entanto, elas não levam em consideração aspectos essenciais do QoE do usuário discutidos neste trabalho. Além disso, os trabalhos atuais em arquiteturas DASH para as futuras gerações redes inteligentes ignoram o comportamento do reprodutor de video em ambientes de Cidades Inteligentes. 
Na continuação deste trabalho, pretendemos trabalhar na implementação mecanismos para melhorar o provisionamento de Streaming de Video em arquiteturas multiníveis. A maneira pela qual o reprodutor de video se comporta para fornecer um delicado equilíbrio entre custo e satisfação do cliente~(em termos de QoE). 
Além disso, um novo cenário de simulação com diferentes domínios entre nós na névoa precisam ser considerados, como realizar a comunicação entre esses domínios em ambientes.


% proposta, e a arquitetura atual do controlador centralizado deve ser
% distribuídos entre agentes locais na direção de uma solução escalável.



% , é possível observar como as soluções existentes podem melhorar

% mas os processos de avaliação de desempenho
% para novas soluções não é uniforme. Algumas soluções não têm validação de desempenho
% enquanto outros trabalhos avaliam suas propostas usando um pequeno banco de testes de software. 

% Através de uma revisão da literatura, é possível observar como as soluções existentes podem melhorar as redes móveis em direção a
% Diferente abordagens são usadas para simplificar o equipamento e aumentar a flexibilidade devido a controles centralizados existentes.

% Existem muitas propostas na área, mas aspectos essenciais parano QoE do usuário não são abordados.

% Algumas soluções não têm validação de desempenho
% enquanto outros trabalhos avaliam suas propostas usando um pequeno banco de testes de software. Para
% Para superar essa necessidade, um novo módulo OpenFlow foi desenvolvido, no ns-3
% simulações nessa área. Além disso, um novo cenário de simulação foi
% proposta, ea arquitetura atual do controlador centralizado deve ser
% distribuídos entre agentes locais na direção de uma solução escalável.

% Tendo em conta que as futuras ambientes em cidades inteligentes organizado hierarquicamente em multiníveis 
% mais células, outro item de interesse nesta pesquisa é como executar
% transferência e transferência de tráfego entre células diferentes e até entre
% diferentes tecnologias. Para isso, propõe-se um mecanismo distribuído para
% lidar com decisões de gerenciamento de mobilidade.


% Através de nosso trabalho, combinamos tecnologias recentes de nevoeiro / nuvem com o estado da arte do ambiente de computação em várias camadas. Identificamos os serviços específicos investigados para o provisionamento de streaming de vídeo. 

% Este trabalho apresentou as motivações de um serviço de streaming de vídeo. Nós descrevemos um conjunto de serviços sob a arquitetura ETSI-NFV. Ele se concentra na demonstração da adequação dos serviços de transcodificador, roteamento de sobreposição, streaming e cache para ambientes de nevoeiro / nuvem de várias camadas. Tentamos caracterizar melhor as propriedades da computação em neblina.